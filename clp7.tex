\clearpage

\section{Esettanulmányok \Clpfd -ben}

Ebben a fejezetben néhány nagyobb \clpfd feladat megoldását ismertetjük, közben
pedig bemutatjuk a \emph{konstruktív diszjunkció}, a \emph{duális címkézés} és a
\emph{borotválás} módszerét.

\subsection{Négyzetdarabolás}

Adott egy nagy négyzet oldalhosszúsága (pl. \cd{Limit = 10}), valamint adottak
kis négyzetek oldalhosszúságai (pl. \cd{Sizes = [6,4,4,4,2,2,2,2]}). A kis
négyzetek területösszege megegyezik a nagy négyzet területével. Meg kell határozni,
hogy le lehet-e fedni a kis négyzetekkel a nagy négyzetet, és ha igen, meg is
kell adni azt, hogy a lefedéshez hova kell helyezni a kis négyzeteket (a nagy
négyzet bal alsó sarka az (1,1) koordinátán van). A példában említett feladat
megoldása: \cd{Xs = [1,7,7,1,5,5,7,9]}, \cd{Ys = [1,1,5,7,7,9,9,9]}.
\br
Források:
\begin{itemize}
\item Pascal van Hentenryck et al. tanulmányának 2. szekciója \\
(\cd{http://www.cs.brown.edu/publications/techreports/reports/CS-93-02.html})
\item SICStus \clpfd példaprogram: \cd{library('clpfd/examples/squares')}
\end{itemize}
Néhány tesztadat:
{\tt
\begin{center}
\begin{tabular}{|r|l|}
\hline
Limit & Sizes \\
\hline
   10 & [6,4,4,4,2,2,2,2]\\
   20 & [9,8,8,7,5,4,4,4,4,4,3,3,3,2,2,1,1]\\
  112 & [50,42,37,35,33,29,27,25,24,19,18,17,16,15,11,9,8,7,6,4,2]\\
  175 & [81,64,56,55,51,43,39,38,35,33,31,30,29,20,18,16,14,9,8,5,4,3,2,1]\\
  503 & [211,179,167,157,149,143,135,113,100,93,88,87,67,62,50,34,33,27, \\
      & 25,23,22,19,16,15,4]\\
\hline
\end{tabular}
\end{center}
}

Az esettanulmány program-változatai, adatai, tesztkörnyezete
megtalálható itt: \cd{http://www.inf.bme.hu/}\verb'~'\cd{szeredi/nlp/nlp_progs_sq.tgz}.
\br
A megoldás során nem foglalkozunk az azonos oldalhosszak miatt jelentkező többszörös
megoldások kiküszöbölésével, mivel az eredeti feladat különböző oldalhosszúságú négyzetekről
szólt, az azonos oldalhosszak csak azért kerültek bele, hogy a programot kisebb
tesztadatokon is tesztelni tudjuk. A teszteseteket minden alkalommal Linux operációs
rendszer alatt egy 600 MHz-es Pentium III gépen futtattuk maximum 120 másodpercig. Ahol
a programvariáns nem adott eredményt 120 másodpercen belül, ott a futási táblázatokban
a mezőt üresen hagytuk. A táblázatokban a futási időt és a visszalépések számát
is feltüntetjük.

\subsubsection{Egyszerű Prolog megoldás}

Az alábbi program Colmerauer \clpr megoldásán alapul, a működési elve hasonlít a
\pageref{teglalap:clpqr}. oldalon található téglalap-lefedő feladathoz.

\begin{verbatim}
% A Limit méretű négyzet lefedhető diszjunkt, Ss oldalhosszúságú
% négyzetekkel, amelyek X és Y kooridnátáit az Xs és Ys lista tartalmazza
squares_prolog(Ss, Limit, Xs, Ys) :-
        triples(Ss, Xs, Ys, SXYs),
        Y0 is Limit+1, 
        XY0 = 1-Y0, 
        NLimit is -Limit,
        filled_hole([NLimit,Limit,Limit], _, XY0, SXYs, []).

% triples(Ss, Xs, Ys, SXYs): SXYs is s(S,X,Y) alakú struktúrákból álló lista
triples([S|Ss], [X|Xs], [Y|Ys], [s(S,X,Y)|SXYs]) :-
        triples(Ss, Xs, Ys, SXYs).
triples([], [], [], []).

% filled_hole(L0, L, XY, SXYs0, SXYs): az L0 vonalban lévő, XY pontban
% kezdődő lyuk az SXYs0 és SXYs listák különbségében lévő négyzetekkel
% lefedve az L vonalat adja
filled_hole(L, L, _, SXYs, SXYs) :-
        L = [V|_], V >= 0, !.
filled_hole([V|HL], L, X0-Y0, SXYs00, SXYs) :-
        V < 0 , Y1 is Y0+V,  
        select(s(S,X0,Y1), SXYs00, SXYs0),
        placed_square(S, HL, L1),  
        Y2 is Y1+S, X2 is X0+S,
        filled_hole(L1, L2, X2-Y2, SXYs0, SXYs1),  
        V1 is V+S,
        filled_hole([V1,S|L2], L, X0-Y0, SXYs1, SXYs).

% placed_square(S, HL, L): a HL vízszintes vonalon az S méretű négyzetet
% elhelyezve az L függőleges vonalat kapjuk
placed_square(S, [H,0,H1|L], L1) :- 
        S > H, !, H2 is H+H1,
        placed_square(S, [H2|L], L1).
placed_square(S, [H,V|L], [X|L]) :- 
        S = H, !, X is V-S.
placed_square(S, [H|L], [X,Y|L]) :- 
        S < H, X is -S, Y is H-S.
\end{verbatim}

\begin{center}
\begin{tabular}{|l|rr|rr|rr|rr|rr|}
\hline
variáns   & 10     &      &  20   &      & 112    &      & 175   &    & 503  &\\
\hline
Prolog          &  0.000 &     0&  0.87& 271K &  0.38 &  183K & 5.72 & 2.6M
&93.58 & 29M \\
\hline
\end{tabular}
\end{center}

\subsubsection{Egyszerű \clpfd megoldás}

Ez a megoldás veszi a kis négyzetek összes koordinátáját, és beállítja
őket úgy, hogy értéküket a nagy négyzeten belül vegyék fel. Ezután
minden négyzet-párra felveszi a \cd{no\_overlap/6} constraintet,
ami azt írja le Prolog választási pontok segítségével, hogy a két
négyzet nem fedi egymást. Végül címkézéssel megadja az eredményt.

\begin{verbatim}
% A feladatra adott egyszerű megoldás spekulatív diszjunkció használatával
squares_spec(Sizes, Limit, Xs, Ys) :-
        generate_coordinates(Xs, Ys, Sizes, Limit),
        state_asymmetry(Xs, Ys, Sizes, Limit),
        state_no_overlap(Xs, Ys, Sizes),
        labeling([], Xs), labeling([], Ys).

% Legenerálja a koordinátákra vonatkozó határokat
generate_coordinates([], [], [], _).
generate_coordinates([X|Xs], [Y|Ys], [S|Ss], Limit) :-
        Sd is Limit-S+1, domain([X,Y], 1, Sd),
        generate_coordinates(Xs, Ys, Ss, Limit).

% Az első négyzet középpontja a bal alsó negyedben van,
% a főátló alatt
state_asymmetry([X|_], [Y|_], [D|_], Limit) :-
        UB is (Limit-D+2)>>1, X in 1..UB, Y #=< X.

% Páronkénti át nem fedést biztosító korlátok felvétele
state_no_overlap([], [], []).
state_no_overlap([X|Xs], [Y|Ys], [S|Ss]) :-
        state_no_overlap(X, Y, S, Xs, Ys, Ss),
        state_no_overlap(Xs, Ys, Ss).

% Megadja, hogy az (X,Y) középpontú, S méretű négyzet nem fedi át a
% többit, amiket a listákban adunk át
state_no_overlap(X, Y, S, [X1|Xs], [Y1|Ys], [S1|Ss]) :-
        no_overlap_spec(X, Y, S, X1, Y1, S1),
        state_no_overlap(X, Y, S, Xs, Ys, Ss).
state_no_overlap(_, _, _, [], [], []).

% no_overlap_spec(X1,Y1,S1, X2,Y2,S2): 
% Az SQ1 = <X1,Y1,S1> négyzet nem fedi át SQ2 = <X2,Y2,S2> -t
% Spekulatív megoldás
no_overlap_spec(X1, _Y1, _S1, X2, _Y2, S2) :-   
        X2+S2 #=< X1.   % SQ1 is above SQ2
no_overlap_spec(X1, _Y1, S1, X2, _Y2, _S2) :- 
        X1+S1 #=< X2.   % SQ1 is below SQ2
no_overlap_spec(_X1, Y1, _S1, _X2, Y2, S2) :-
        Y2+S2 #=< Y1.   % SQ1 is to the right of SQ2
no_overlap_spec(_X1, Y1, S1, _X2, Y2, _S2) :-   
        Y1+S1 #=< Y2.   % SQ1 is to the left of SQ2
\end{verbatim}

Ezzel a megoldással sokkal kevésbé hatékony változatot kaptunk:

\begin{center}
\begin{tabular}{|l|rr|rr|rr|rr|rr|}
\hline
variáns   & 10     &      &  20   &      & 112    &      & 175   &    & 503  &\\
\hline
Prolog          &  0.000 &     0&  0.87& 271K &  0.38 &  183K & 5.72 & 2.6M & 93.58 & 29M \\
\cd{spec}       &  1.99  &   34K&       &     &       &     &       &        &       &\\
\hline
\end{tabular}
\end{center}

Érdemes megfigyelni, hogy míg a Prolog megoldás a 10-es feladatot visszalépés nélkül oldotta
meg, addig ehhez a \cd{spec} változatnak 34 ezer visszalépésre volt szüksége. A problémát
a \cd{no_overlap/6} korlát spekulatív diszjunkcióval való megvalósítása jelenti, hiszen
ez a kombinatorikus robbanás miatt nagyon sok visszalépést eredményez.

\subsubsection{A diszjunkció megvalósítási módszerei}

\label{diszjunkcio}

Az előző variánsban gondot okozó spekulatív diszjunkció kiváltására több lehetőségünk is
van, ezeket egy egyszerűbb példán fogjuk megnézni. Legyen ez az egyszerűbb példa a
következő:

\begin{verbatim}
| ?- domain([X,Y], 0, 6), ( X+5 #=< Y ; Y+5 #=< X).
X in 0..1, Y in 5..6 ? ;
X in 5..6, Y in 0..1 ? ;
no
\end{verbatim}

A diszjunkció megvalósítására kézenfekvő megoldás, ha tükrözést használunk:

\begin{verbatim}
| ?- domain([X,Y], 0, 6), X+5 #=< Y #\/ Y+5 #=< X.
X in 0..6, Y in 0..6 ? ;
no
\end{verbatim}

Amint látjuk, ennek az a hátránya, hogy nem hajt végre maximális szűkítést, nem jön
rá, hogy \cd{X} és \cd{Y} semmiképpen nem lehet 2, 3 vagy 4. A korlát-megoldó ugyanis
egy diszjunkció esetén csak akkor tud tenni valamit, ha a diszjunkcióban részt vevő
korlátok közül egyet kivéve már az összesről eldőlt, hogy nem állhat fenn. Érdemes
tehát további megoldásokon gondolkoznunk. A jelen esetben például kikerülhetjük a
diszjunkciót az \cd{abs} korlát használatával:

\begin{verbatim}
| ?- domain([X,Y], 0, 6), 'x+y=t tsz'(Y, D, X), abs(D) #>= 5.
X in(0..1)\/(5..6), Y in(0..1)\/(5..6) ? ;
no
\end{verbatim}

Ezt azonban nagyon sok esetben nem tehetjük meg. Átírhatjuk viszont a diszjunkciót
indexikálissá:

\begin{verbatim}
ix_disj(X, Y) +: 
        X in \(max(Y)-4..min(Y)+4), Y in \(max(X)-4..min(X)+4).

| ?- ix_disj(X, Y).
X in(0..1)\/(5..6), Y in(0..1)\/(5..6) ? ;
no
\end{verbatim}

Sajnos az indexikálisokra fennálló korlátozások (pl. fix számú változó) miatt ezt
is csak speciális esetekben tehetjük meg. Most egy általános szűkítési módszert,
a \emph{konstruktív diszjunkció}t fogjuk ismertetni.
\br
A konstruktív diszjunkció alapötlete a következő: a diszjunkció minden tagja esetén
vizsgáljuk meg a hatását a tárra, és jelöljük az így kapott ,,vagylagos'' tárakat
$S_1, \ldots, S_n$-nel. Ekkor minden változó a vagylagos tárakban kapott tartományok
uniójára szűkíthető: \cd{X in_set $\cup D({\tt X}, S_i)$}. A konstruktív diszjunkciót
általánosan az alábbihoz hasonló módon lehet megvalósítani:

\begin{verbatim}
cdisj(Cs, Var) :-
        empty_fdset(S0), cdisj(Cs, Var, S0, S), 
        Var in_set S.

cdisj([Constraint|Cs], Var, Set0, Set) :-
        findall(S, (Constraint,fd_set(Var,S)), Sets),
        fdset_union([Set0|Sets], Set1),
        cdisj(Cs, Var, Set1, Set).
cdisj([], _, Set, Set).

| ?- domain([X,Y], 0, 6), cdisj([X+5 #=< Y,Y+5 #=< X], X).
X in(0..1)\/(5..6), Y in 0..6 ? 
\end{verbatim}

A konstruktív diszjunkció akár erősebb is lehet, mint a tartomány-szűkítés, mert más
korlátok hatását is figyelembe tudja venni, lásd az alábbi példát:

\begin{verbatim}
| ?- domain([X,Y], 0, 20), X+Y #= 20, cdisj([X#=<5,Y#=<5],X).
X in(0..5)\/(15..20), Y in(0..5)\/(15..20) ? 
\end{verbatim}

\subsubsection{\clpfd megvalósítás reifikációval és indexikálissal}

\enumhead{Számosság-alapú \cd{no\_overlap} változatok}

\begin{verbatim}
no_overlap_card1(X1, Y1, S1, X2, Y2, S2) :-
        X1+S1 #=< X2 #<=> B1,
        X2+S2 #=< X1 #<=> B2,
        Y1+S1 #=< Y2 #<=> B3,
        Y2+S2 #=< Y1 #<=> B4,
        B1+B2+B3+B4 #>= 1.

no_overlap_card2(X1, Y1, S1, X2, Y2, S2) :-
  call( abs(2*(X1-X2)+(S1-S2)) #>= S1+S2 #\/
        abs(2*(Y1-Y2)+(S1-S2)) #>= S1+S2 ).
\end{verbatim}

\enumhead{Indexikális \cd{no\_overlap} (,,gyenge'' konstruktív diszjunkció)}
\br
Alapgondolat: Ha két négyzet Y irányú vetületei biztosan átfedik egymást,
akkor X irányú vetületeik diszjunktak kell legyenek, és fordítva. Az Y irányú
vetületek átfedik egymást, ha mindkét négyzet felső széle magasabban van mint
a másik négyzet alsó széle: \cd{Y1+S1>Y2} és \cd{Y2+S2>Y1}. Indexikálisban ezt
a következőképpen tudjuk megfogalmazni: ha a \cd{(Y1+S1..Y2) \bs/ (Y2+S2..Y1)} halmaz
üres, akkor a fenti feltétel fennáll, tehát X irányban szűkíthetünk: \cd{X1 =< X2-S1}
vagy \cd{X1 >= X2+S2}. Feltételes kifejezéssel:
\begin{center}
\cd{X1 in ((Y1+S1..Y2)\bs/(Y2+S2..Y1))?(inf..sup) \bs/ \bs(X2-S1+1..X2+S2-1)}
\end{center}
Ezen ötlet felhasználásával az indexikális:

\begin{alltt}
no_overlap_ix(X1, Y1, S1, X2, Y2, S2) +:
{\tt\em{}%       ha Y irányú átfedés van, azaz}
{\tt\em{}%       ha min(Y1)+S1 > max(Y2) és min(Y2)+S2 > max(Y1) ...}
        X1 in ((min(Y1)+S1..max(Y2)) \bs/ (min(Y2)+S2..max(Y1)))
{\tt\em{}%                       ... akkor X irányban nincs átfedés: }
             ? (inf..sup) \bs/ \bs(max(X2)-(S1-1) .. min(X2)+(S2-1)),
        X2 in ((min(Y1)+S1..max(Y2)) \bs/ (min(Y2)+S2..max(Y1)))
             ? (inf..sup) \bs/ \bs(max(X1)-(S2-1) .. min(X1)+(S1-1)),
        Y1 in ((min(X1)+S1..max(X2)) \bs/ (min(X2)+S2..max(X1)))
             ? (inf..sup) \bs/ \bs(max(Y2)-(S1-1) .. min(Y2)+(S2-1)),
        Y2 in ((min(X1)+S1..max(X2)) \bs/ (min(X2)+S2..max(X1)))
             ? (inf..sup)\bs/ \bs(max(Y1)-(S2-1) .. min(Y1)+(S1-1)).
\end{alltt}

\begin{center}
\begin{tabular}{|l|rr|rr|rr|rr|rr|}
\hline
variáns   & 10     &      &  20   &      & 112    &      & 175   &    & 503  &\\
\hline
Prolog     &0.00&    0&  0.87& 271K &  0.38 &  183K & 5.72 & 2.6M & 93.58 & 29M \\
\cd{spec}  &1.99&  34K&       &     &       &     &       &        &       &\\
\cd{card1} &0.07&  141&       &     &       &     &       &       &       &     \\
\cd{card2} &0.07&  141&       &     &       &     &       &       &       &     \\
\cd{ix}    &0.01&  141&       &     &       &     &       &       &       &     \\
\hline
\end{tabular}
\end{center}

A visszalépések száma lényegesen javult a spekulatív diszjunkcióval kapott változathoz
képest, azonban a program még így is nyomába sem ér a hagyományos Prolog megvalósításnak.

\subsubsection{Kapacitás-korlátok és paraméterezhető címkézés}

Mint azt már említettük, lehetőség van a címkézés menetének befolyásolására a
\cd{labeling/2} eljárás paraméterlistáján keresztül. Az eredeti Prolog megoldás
a ,,tetris-elv'' szerint alulról felfelé töltötte fel a nagy négyzetet a
kis négyzetekkel, ennek egy elég jó megközelítése a \cd{[min,step]} üzemmódú
címkézés. Próbálgatás céljából esetleg érdemes külön paraméterezhetővé tenni
a címkézési módokat, majd összevetni az egyes variánsokat.
\br
További gyorsítás érhető el redundáns korlátok használatával. A jelenlegi program
ugyanis még nem elég okos: például amikor a nagy négyzet alja betelt, nem hagyja
ki az Y változók tartományából az 1 értéket, pedig oda már biztosan nem tudunk
további négyzeteket rakni. Az ún.\ kapacitás-korlátokkal ez megvalósítható:
ha összeadjuk azon kis négyzetek oldalhosszát, amelyek elmetszenek egy X=1, X=2,
\ldots, Y=1, Y=2, \ldots vonalat, akkor a nagy négyzet oldalhosszát kell kapnunk
(a kis négyzeteket itt alulról és balról zártnak, felülről és jobbról nyíltnak
tekintjük). Például X irányban:

$$\sum  \{ \cd{S}_i | p \in [\cd{X}_i,\cd{X}_i+\cd{S}_i) \} =
\cd{Limit} \ \ \ ( \forall p \in \cd{1..Limit-1} )$$

A kapacitás-korlátok az alábbi kódrészlettel felvehetőek:

\begin{verbatim}
% A feladatra adott egyszerű megoldás kapacitás-korlátok használatával
squares_cap(Lab, Sizes, Limit, Xs, Ys) :-
        generate_coordinates(Xs, Ys, Sizes, Limit),
        state_asymmetry(Xs, Ys, Sizes, Limit),
        state_no_overlap(Xs, Ys, Sizes),
        state_capacity(1, Xs, Sizes, Limit),
        state_capacity(1, Ys, Sizes, Limit),
        labeling(Lab, Xs), labeling(Lab, Ys).
\end{verbatim}
\begin{verbatim}
% Kapacitás-korlát a Cs koordinátákra Sizes oldalhosszúságú
% kis négyzetek és Limit méretű nagy négyzet esetén a
% Pos..Limit intervallum összes elemére
state_capacity(Pos, Limit, Cs, Sizes) :-
        Pos =< Limit, !, accumulate(Cs, Sizes, Pos, Bs),
        scalar_product(Sizes, Bs, #=, Limit), 
        Pos1 is Pos+1, state_capacity(Pos1, Limit, Cs, Sizes).
state_capacity(_Pos, _Limit, _, _).
\end{verbatim}
\begin{alltt}
% accumulate(C, S, Pos, B): B, C és S ugyanolyan hosszú listák,
% \cd{B}\(_i\)-k B elemei, \(\cd{B}_i=1 \Leftrightarrow \cd{Pos} \in [\cd{C}_i,\cd{C}_i+\cd{S}_i)\),
accumulate([], [], _, []).
accumulate([Ci|Cs], [Si|Ss], Pos, [Bi|Bs]) :-
        Crutch is Pos-Si+1, Ci in Crutch .. Pos #<=> Bi,
        accumulate(Cs, Ss, Pos, Bs).
\end{alltt}

\begin{center}
\begin{tabular}{|l|rr|rr|rr|rr|rr|}
\hline
variáns, címkézés  & 10     &      &  20   &      & 112    &      & 175   &    & 503  &\\
\hline
Prolog     &0.00&    0&  0.87& 271K &  0.38 &  183K & 5.72 & 2.6M & 93.58 & 29M \\
\cd{[]-ix, [min]}     &  0.01&   84&       &     &       &     &       &     &       &     \\
\cd{cap-ix, []}       &  0.01&    0&  0.07&   18&       &     &       &     &       &     \\
\cd{cap-ix, [min]}    &  0.01&    0&  0.06&    0&  1.96&  109&  3.74&  105& 20.32&  405\\
\hline
\cd{cap-spec, [min]}  &  2.31&34K&       &     &       &     &       &     &       &     \\
\cd{cap-card1, [min]} &  0.04&    0&  0.24&    0&  3.51&  109&  4.86&  105& 22.63&  405\\
\cd{cap-card2, [min]} &  0.04&    0&  0.34&    0&  2.41&  109&  4.48&  105& 21.83&  405\\
\hline
\end{tabular}
\end{center}

Amint látható, a kapacitás-korlátokkal a nagyobb méretekre a \clpfd megoldás már
lekörözi a hagyományos Prolog megoldást.

\subsubsection{Ütemezési és lefedési korlátok használata}

A négyzetdarabolás felfogható ütemezési problémaként, illetve diszjunkt téglalapok
problémájaként is. Mindkét feladatra van beépített korlát a SICStusban. Ütemezési
probléma esetén alkalmazhatjuk a \cd{cumulative/5} korlátot mindkét tengely irányában,
diszjunkt téglalapok problémája esetén pedig a \cd{disjoint2/2} korlátot (ilyenkor a
\cd{no\_overlap} használatától akár el is tekinthetünk).
\br
A két újabb változat:

\begin{verbatim}
squares_cum(Lab, Opts, Sizes, Limit, Xs, Ys) :-
        generate_coordinates(Xs, Ys, Sizes, Limit),
        state_asymmetry(Xs, Ys, Sizes, Limit),
        state_no_overlap(Xs, Ys, Sizes),   
        cumulative(Xs, Sizes, Sizes, Limit, Opts),
        cumulative(Ys, Sizes, Sizes, Limit, Opts),
        labeling(Lab, Xs), labeling(Lab, Ys).

squares_dis(Lab, Opts, Sizes, Limit, Xs, Ys) :-
        generate_coordinates(Xs, Ys, Sizes, Limit),
        state_asymmetry(Xs, Ys, Sizes, Limit),
        state_no_overlap(Xs, Ys, Sizes),    % ez elmarad a`none' 
                                            % variáns esetén
        disjoint2_data(Xs, Ys, Sizes, Rects),
        disjoint2(Rects, Opts),
        labeling(Lab, Xs), labeling(Lab, Ys).

disjoint2_data([], [], [], []).
disjoint2_data([X|Xs], [Y|Ys], [S|Ss], [r(X,S,Y,S)|Rects]) :-
        disjoint2_data(Xs, Ys, Ss, Rects).
\end{verbatim}

Az alábbi tesztek során mindig \cd{[min]} címkézést használtunk, és a globális
korlátok paraméterezésével variáltunk. Rövidítések: \cd{e = edge_finder(true)},
\cd{g = global(true)}.

\begin{center}
\begin{tabular}{|l|rr|rr|rr|rr|rr|}
\hline
variáns   & 10     &      &  20   &      & 112    &      & 175   &    & 503  &\\
\hline
{\tt cum-ix}      &  0.00&    0&  0.02&    0&       &     &       &     &       &     \\
{\tt cum(e)-ix}   &  0.01&    0&  0.01&    0&  0.18&  139&  0.12&   67&  0.52&  421\\
\hline
{\tt dis-none}    &  0.01&   52&       &     &       &     &       &     &       &     \\
{\tt dis(g)-none} &  0.00&    0&  0.01&    0&  0.73&  282&  0.41&  133&  2.55&  576\\
{\tt dis(g)-ix}   &  0.00&    0&  0.02&    0&  0.93&  282&  0.53&  133&  2.95&  576\\
\hline
\end{tabular}
\end{center}

\subsubsection{Duális címkézés}

Duális címkézés során nem a változókhoz keresünk megfelelő értéket, hanem az értékekhez
megfelelő változót. Kicsit formálisabban fogalmazva az algoritmus lényege:

\begin{itemize}
\item vegyük sorra a lehetséges változó-értékeket,
\item egy adott $e$ értékhez keresünk egy $V$ változót, amely felveheti ezt
az értéket,
\item csináljunk egy választási pontot: $V = e$, vagy $V \neq e$, stb.
\end{itemize}

Növekvő értéksorrend esetén a keresési tér meg fog egyezni a \cd{[min, step]} beépített
címkézés keresési terével.

\begin{verbatim}
% dual_labeling(L, Min, Max): címkézi az L lista elemeit, ahol
% minden L-beli X változóra X in Min..Max fennáll.
% A programban használt hívási formátum:
% dual_labeling(Xs,1,Limit),dual_labeling(Ys,1,Limit).
dual_labeling([], _, _) :- !.
dual_labeling(L0, Min0, Limit) :-
        dual_labeling(L0, L1, Min0, Limit, Min1),
        dual_labeling(L1, Min1, Limit).

% dual_labeling(L0, L, I, Min0, Min): címkézi az L0 lista változóit
% az I értékkel, amikor ez lehetséges, a maradék változókat L-ben adja
% vissza. Ugyanakkor Min0-Min-ben gyűjti az L-beli változók
% alsó határát
dual_labeling([], [], _, Min, Min).
dual_labeling([X|L0], L, I, Min0, Min) :-
        (   integer(X) -> dual_labeling(L0, L, I, Min0, Min)
        ;   X = I, 
            dual_labeling(L0, L, I, Min0, Min)
        ;   X #> I, 
            fd_min(X, Min1), Min2 is min(Min0,Min1),
            L = [X|L1], dual_labeling(L0, L1, I, Min2, Min)
        ).
\end{verbatim}

\begin{center}
\begin{tabular}{|l|rr|rr|rr|rr|rr|}
\hline
variáns; címkézés   & 10     &      &  20   &      & 112    &      & 175   &    & 503  &\\
\hline
\cd{cum(e)-ix; [min]}    &  0.01&    0&  0.01&    0&  0.18&  139&  0.12&   67&  0.52&  421\\
\cd{cum(e)-ix; dual}     &  0.01&    0&  0.02&    0&  0.19&  139&  0.13&   67&  0.54&  421\\
\hline
\cd{cap-cum(e)-ix;}      &  0.02&    0&  0.07&    0&  1.77&  100&  3.22&   65& 17.26&  395\\
\cd{cap-dis(g)-none;}    &  0.01&    0&  0.06&    0&  1.71&   97&  3.24&   66& 17.98&  393\\ 
\cd{cum(e),dis(g)-none;} &  0.00&    0&  0.01&    0&  0.23&  136&  0.16&   67&  0.99&  419\\
\hline
\end{tabular}
\end{center}




\subsection{Torpedó}

Adott egy téglalap alakú táblázat, amelyben $1 \times n$-es hajókat kell elhelyezni
úgy, hogy még átlósan se érintkezzenek. A hajók különböző színűek lehetnek. Minden
szín esetén adott:
\begin{itemize}
\item minden hajóhosszhoz: az adott színű és hosszú hajók száma;
\item minden sorra és oszlopra: az adott színű hajó-darabok száma;
\item ismert hajó-darabok a táblázat mezőiben.
\end{itemize}
Színfüggetlenül adottak az ismert torpedó-mentes (tenger) mezők.
\br
{\bf Példa:} Két szín, mindkét színből 1 darab egyes és 1 darab kettes hajó.
Ismert mezők: az 1. sor 1. mezője tenger, az első sor 3. mezője egy kettes
hajó tatja (jobb vége).

\begin{alltt}
     {\rm \em A feladat:}                                     {\rm \em A megoldás:}
\end{alltt}
\begin{alltt}
      1 2 3 4 5     <-- {\em oszlopszám     }              1 2 3 4 5     
      0 1 1 1 0     <-- {\em 1. oszlopössz. }              0 1 1 1 0     
                        {\em                }                            
1  2  =   r      0      {\em                }        1  2  = * r : :  0
2  0             1      {\em                }        2  0  : : : : #  1  
3  0             1      {\em                }        3  0  # : : : :  1  
4  1             1      {\em                }        4  1  # : : * :  1  
   ^-------------^------{\em sorösszegek    }          
      2 0 0 0 1     <-- {\em 2. oszlopössz. }              2 0 0 0 1     
\end{alltt}

A fenti példában alkalmazott jelölésrendszer: a tábla felett az első számsor az
oszlopok számozását jelenti, a tábla mellett az első oszlop a sorok számozását.
Közvetlenül a tábla szélei mellett lévő számsorok az adott sorban, illetve oszlopban
lévő hajódarabkák számát adják meg, a felső, illetve a bal oldali az 1. színét,
az alsó, illetve a jobb oldali pedig a 2. színét. A táblában a tengert \cd{=}
(egyenlőségjel) karakterrel, a hajódarabkákat pedig betűkkel jelöljük. Az 1. szín
hajódarabkáit kisbetűkkel, a 2. színét nagybetűkkel ábrázoljuk. Az 1 hosszú hajók
\cd{o}, illetve \cd{O} betűkkel vannak jelölve, a hosszabb hajók esetén pedig \cd{u}
(\cd{U}) a hajó felső vége, \cd{d} (\cd{D}) az alsó vége, \cd{l} (\cd{L}) a bal széle,
\cd{r} (\cd{R}) a jobb széle, \cd{m} (\cd{M}) pedig a közepe. A kikövetkeztetett
hajódarabkákat \cd{*} (csillag) és \cd{\#} (hashmark) karakterek jelölik, a
kikövetkeztetett tengerdarabkákat pedig \cd{:} (kettőspont).
\br
A feladat az 1999. évi NLP kurzus nagyházifeladata volt, a mintamegoldás letölthető
az alábbi címről: \verb'http://www.inf.bme.hu/~szeredi/hf_99_torpedo.tgz'.

\subsubsection{A feladat modellezése}

A feladat komplexitásából kifolyólag eleve érdemes azon elgondolkoznunk, hogy hogyan
feleltessük meg a feladatot a CLP világnak. Például a korlát-változók felvételére
is eleve két lehetőségünk van:

\begin{itemize}
\item[a.] Minden hajóhoz hozzárendelünk egy irányváltozót (vízszintes vagy
függőleges) és a kezdőpont koordinátáit. Így viszonylag kevés változóval
,,megússzuk'', cserébe viszont szimmetria problémák léphetnek fel
(azonos méretű hajók sorrendje), a korlátjaink bonyolultabbak lesznek és
sok diszjunktív korlátot kell alkalmaznunk (pl. egy hajó vízszintes vagy
függőleges elhelyezés esetén más-más mezőket fed le).
\item[{\bf b.}] A mezőkhöz rendelünk változókat, és mezőnként tároljuk, hogy mi
található ott: hajó-darab vagy tenger. Ezzel ugyan több változónk lesz,
de a korlátok lényegesen leegyszerűsödnek, ezért ezt a megoldást választjuk.
\end{itemize}

Ezek után el kell gondolkoznunk azon, hogy az egyes mezőkhöz tartozó változóknak
milyen értékkészletet adunk. Újfent két, lényegében eltérő megoldás között
választhatunk:

\begin{itemize}
\item[a.] egy mezőről csak azt tároljuk, hogy hajódarab vagy pedig tenger van ott,
a hajódarabról pedig a színt is megjegyezzük. Mivel nem rögzítjük, hogy egy hajódarab
a hajó melyik részét alkotja, ezért az eleve ismert mezőknél információvesztés lép fel.
\item[{\bf b.}] az egyes hajódarabokat is megkülönböztetjük:
\begin{itemize}
\item[b1.] az előre kitöltött mezőknek megfelelő darabok \cd{(u,l,m,r,d,o)} --- ilyenkor
ismét diszjunktív korlátokat kell alkalmazni (pl.\ ugyanaz a betű többféle hajó része
lehet)
\item[{\bf b2.}] részletesebb bontás: a mezőket megkülönböztetjük a hajó hossza, iránya, a
darab hajón belüli pozíciója szerint, pl.: egy 4 hosszú vízszintes hajó balról
3. darabja. A megoldásban ezt a módszert alkalmazzuk.
\end{itemize}
\end{itemize}

Vegyük észre, hogy a választott megoldás jellemzője az, hogy ha egy mezőhöz tartozó
változó tengertől különböző értéket kap, akkor ezzel már az egész hajót meghatároztuk.
\br
Mivel egy mezővel kapcsolatban több információt is rögzíteni akarunk, felvetődik a kérdés,
hogy külön változókkal adjuk meg az egyes jellemzőket, vagy pedig egyetlen változóban
jelenítsük meg az összeset. Az első esetben nyilvánvalóan egyszerűbb lesz a kódolás,
de a korlátok szűkítései gyengébbek lesznek, mivel egy korlátnak nem feltétlenül
áll rendelkezésére az összes információ. A második esetben ugyan bonyolultabb kódolást
kell megvalósítani, de cserébe a korlátok erősebben szűkítenek majd, és mivel a
megoldásunkban elsősorban a sebesség a lényeg, ezért ehhez a módszerhez érdemes folyamodni.
\br
Összefoglalva a választott irányelveket:

\begin{itemize}
\item Minden mezőnek egy változó felel meg.
\item Az értékek kódolási elvei (\cd{max} címkézéshez igazítva)
\begin{itemize}
\item az irányított hajók orra (\cd{l} és \cd{u}) kapja a legmagasabb
kódokat,
\item ezen belül a hosszabbak kapják a nagyobb kódokat
\item adott hossz esetén az irány és a szín sorrendje nem fontos
\item az irányított hajók nem-orr elemeinek kódolása nem lényeges (címkézéskor
az orr-elemek helyettesítődnek be)
\item az egy hosszú hajók (hajódarabok) kódja a legalacsonyabb
\item a tenger kódja minden hajónál alacsonyabb
\end{itemize}
\item Példa-kódolás: 1 szín, max 3 hosszú hajók, \cd{h$ij$} = horizontális (vízszintes),
$i$ hosszú hajó $j$-edik darabja, \cd{v$ij$} = vertikális (függőleges) hajó megfelelő
darabja, stb. A kódkiosztás: 

\begin{verbatim}
0:       tenger
1:       h11 = v11      % 1-hosszú hajó
2..4     v33  h22 h32   % nem-orr-elemek
5..7     v32 v22  h33   % nem-orr-elemek
8..9     h21 v21        % orr-elemek
10..11   h31 v31        % orr-elemek
\end{verbatim}
\end{itemize}

\subsubsection{Alapvető korlátok}

A feladat megoldásához két alapvető korlátra lesz szükségünk:

\begin{itemize}
\item \cd{coded\_field\_neighbour(Dir, CF0, CF1)}: \cd{CF0} kódolt mező
\cd{Dir} irányú szomszédja \cd{CF1}, ahol \cd{Dir} lehet \cd{horiz,
vert, diag}. Például:

\begin{verbatim}
| ?- coded_field_neighbour(horiz, 0, R).
R in \{3,4,7} ? ;
no
\end{verbatim}

\item \cd{group\_count(Group, CFs, Count, Env):} a \cd{Group} csoportba
tartozó elemek száma a \cd{CFs} listában \cd{Count}, ahol a futási
környezet \cd{Env}. Itt \cd{Group} például lehet \cd{all(Clr)}: az összes
\cd{Clr} színű hajódarab. Ez a \cd{count/4} eljárás kiterjesztése: nem
egyetlen szám, hanem egy számhalmaz előfordulásait számoljuk meg.
\end{itemize}

Ezen korlátokat az ismert mezők megfelelő csoportokra való megszorítása
után a következőképpen kell felvenni:

\begin{enumerate}
\item Színenként az adott sor- és oszlopszámlálók előírása (erre jó az
előző részfejezetben felvázolt \cd{group\_count/4} predikátum).
\item A hajóorr-darabok megszámolásával az adott hajófajta darabszámának
biztosítása (\cd{group\_count/4}, minden színre és minden hajófajtára).
\item A vízszintes, függőleges és átlós irányú szomszédos mezőkre vonatkozó
korlátok biztosítása a \cd{coded\_field\_neighbour/3} korláttal.
\end{enumerate}

A 2. fajtájú korlátoknál a részösszegekre néhol érdemes segédváltozókat
bevezetni (pl.\ \cd{A+B+C \#= 2, A+B+D \#= 2} helyett \cd{A+B \#= S, S+C
\#= 2, S+D \#= 2} jobban tud szűkíteni, mert az \cd{S} változón keresztül a
két összegkorlát ,,kommunikál'' egymással). Formálisan: jelölje $sor^K_s$
ill.\ $oszl^L_s$ az $s$ hajódarab előfordulási számát a $K$-adik sorban,
ill.\ az $L$-edik oszlopban. A hajók számolásához a $sor^K_{\tt hI1}$ és
$oszl^L_{\tt vI1}$ mennyiségekre segédváltozókat vezetünk be, ezekkel a 3. korlát:

\begin{center}
az \cd{I} hosszú hajók száma = 
$\sum_{K} sor^K_{\tt hI1} + \sum_{L} oszl^L_{\tt vI1} \ \ \ \ ({\tt I} > 1)$
\end{center}

\begin{center}
az 1 hosszú hajók száma = $\sum_{K} sor^K_{\tt h11} $
\end{center}

\subsubsection{Redundáns korlátok, címkézés és borotválás}

A fenti alapvető korlátok mellé még az alábbi redundáns korlátokat érdemes
bevezetni:

\begin{itemize}
\item \cd{count\_ships\_occs}: sorösszegek alternatív kiszámolása (vö.\ a
mágikus sorozatok megoldásában a skalárszorzat redundáns korláttal):
\begin{center}
a $K$. sorbeli darabok száma = \parbox{19em}{
{\large \[\sum_{\cd{I}\leq hosszak} \cd{I}*sor^K_\cd{hI1} + \sum_{1<\cd{I}\leq hosszak, \cd{J}\leq \cd{I}} sor^K_\cd{vIJ}\]}
}\end{center}
Analóg módon az oszlopösszegekre is.

Ennek a korlátnak a hatására ,,veszi észre'' a program, hogy  ha pl. egy
sorösszeg 3, akkor nem lehet a sorban 3 eleműnél hosszabb hajó.

\item \cd{count\_ones\_columns}: az egy hosszú darabok számát az
oszloponkénti előfordulások összegeként is meghatározzuk.
\item \cd{count\_empties}: minden sorra és oszlopra a tenger-mezők
számát is előírjuk (a sorhosszból kivonva az összes --- különböző színű ---
hajódarab összegét).
\end{itemize}

A mintamegoldásban az alábbi címkézési fajták vannak implementálva
(\cd{label(\emph{Variáns})} opciók):

\begin{itemize}
\item \cd{plain}: \cd{labeling([max,down], Mezők)}, ahol \cd{Mezők} a mezőváltozókat
        tartalmazó lista.
\item \cd{max\_dual}: a négyzetkirakáshoz hasonlóan a legmagasabb
        \emph{értékeket} próbálja a változóknak értékül adni. Ez szűkítő
        hatásban (és így a keresési fa szerkezetében) azonos a \cd{plain}
        variánssal.
\item \cd{ships}: speciális címkézés, minden hosszra, a legnagyobbtól
        kezdve, minden színre az adott színű és hosszú hajókat sorra elhelyezi
        (ez az alapértelmezés).
\end{itemize}

A megoldás a konstruktív diszjunkció egy egyszerűsített verzióját, a \emph{borotválás}t
is használja a címkézés során. A borotválás lényege az, hogy minden $n$. címkézési
lépésben a címkézésből hátralévő változók mindegyikét megpróbáljuk egy adott
tartományra (jelen esetben ,,tenger''-re) helyettesíteni, és ha ez azonnal meghiúsulást
okoz, akkor a kipróbált tartományt kizárhatjuk a változó tartományából (jelen esetben
megállapíthatjuk, hogy azon a mezőn hajódarab van). A módszert $n$ változtatásával
és a tartomány megválasztásával lehet ,,finomhangolni''. A torpedó feladatban alkalmazott
borotválást minden szín címkézése előtt megismételjük. A \cd{filter(\emph{VariánsLista})}
opción keresztül változtathatjuk a borotválás jellegét: ha a lista eleme \cd{off},
akkor nincs borotválás, \cd{on} esetén egyszeres borotválás van, \cd{repetitive} esetén
pedig minden borotválásnál ismételten szűrünk addig, amíg az újabb korlátokat eredményez.
\br
A borotválás egy lépését az alábbi programkóddal végezhetjük el:
\begin{verbatim}
% filter_count_vars(Vars0, Vars, Cnt0, Cnt): Vars0 megszűrve
% Vars-t adja. A megszűrt változók száma Cnt-Cnt0.
filter_count_vars([], [], Cnt, Cnt).
filter_count_vars([V|Vs], Fs, Cnt0, Cnt) :-
        integer(V), !, filter_count_vars(Vs, Fs, Cnt0, Cnt).
filter_count_vars([V|Vs], [V|Fs], Cnt0, Cnt) :-
        (   fd_min(V, Min), Min > 0 -> Cnt1 = Cnt0
        ;   \+ (V = 0) -> V #\= 0, Cnt1 is Cnt0+1
        ;   Cnt1 = Cnt0
        ), filter_count_vars(Vs, Fs, Cnt1, Cnt).
\end{verbatim}

\subsubsection{További finomhangolási lehetőségek}

A szomszédsági reláció megvalósítására több lehetőség is kínálkozik:

\begin{itemize}
\item A vízszintes és függőleges szomszédsági reláció egyaránt megvalósítható
a \cd{relation/3} meghívásával vagy indexikálisként való fordításával is.
Ezek között az opciólistában a \cd{relation(R)} elemmel kapcsolgathatunk
(\cd{R = clause} vagy \cd{R = indexical}). Alapértelmezésként a korlát
indexikálisként van megvalósítva.

\item Az átlós szomszédsági relációt is többféleképpen megvalósíthatjuk.
A kívánt variánst a \cd{diag(D)} opcióval választhatjuk ki, ahol \cd{D}
lehet:

\begin{itemize}
\item \cd{reif} --- reifikációs alapon: \cd{CF1 \#= 0 \#\bs/ CF2 \#= 0}
\item \cd{ind\_arith} --- aritmetikát használó indexikálissal:\\
\cd{\ \ \ diagonal\_neighbour\_arith(CF1, CF2) +:\\
\ \ \ \ \ \ CF1 in 0 .. (1000-(min(CF2)/>1000)*1000), ...}
\item \cd{ind\_cond} (alapértelmezés) --- feltételes indexikálissal:
\cd{\ \ \ diagonal\_neighbour\_cond(CF1, CF2) +:\\
\ \ \ \ \ \ CF1 in (min(CF2)..0) ? (inf..sup) \bs/ {0}, ...} 
\end{itemize}
\end{itemize}

\subsubsection{Futási eredmények}

Az alábbi időeredmények az összes megoldás megtalálására vonatkoznak, és egy
DEC Alpha 433 MHz-es gépen születtek. A táblázatokban lévő adatpárok a futási
időt (mp) és a visszalépések számát jelentik.

\begin{center}
\begin{tabular}{|l|rr|rr|rr|}
\hline
Opciók/példa   & \multicolumn{2}{l|}{\cd{fules2a}}      & \multicolumn{2}{l|}{\cd{fules3}}  & \multicolumn{2}{l|}{\cd{fules\_clean}}   \\
\hline
1. sima                               &   51.437 &    10178 &  253.1 &    55157 & 1085.7 &     260K\\
\hline
\multicolumn{7}{|l|}{Redundáns korlátok}\\
\hline
2. = 1 $+$ \cd{count_ships_occs     }  &   16.218 &     1910 &  105.6 &    13209 &  395.2 &    52398\\
3. = 2 $+$ \cd{count_ones_columns   }  &   16.175 &     1861 &  105.0 &    12797 &  386.4 &    50181\\
4. = 3 $+$ \cd{count_empties        }  &   17.915 &     1771 &  107.2 &    11273 &  381.7 &    42417\\
\hline
\multicolumn{7}{|l|}{Címkézési variánsok}\\
\hline
5. = 4 $+$ \cd{label(max_dual)      }  &   18.296 &     1771 &  106.3 &    11273 &  379.8 &    42417\\
6. = 4 $+$ \cd{label(ships)         }  &   17.153 &     1708 &  105.7 &    11236 &  367.8 &    41891\\
\hline
\multicolumn{7}{|l|}{Borotválás}\\
\hline
7. = 6 $+$ \cd{filter([repetitive])}   &   10.517 &      313 &   64.3 &     2534 &  206.1 &    10740\\
8. = 6 $+$ \cd{filter([on])         }  &    9.549 &      332 &   59.0 &     2811 &  199.7 &    12004\\
\hline
\multicolumn{7}{|l|}{Megvalósítási variánsok}\\
\hline
9. = 8 $+$ \cd{relation(indexical)}    &    8.426 &      332 &   54.0 &     2811 &  180.8 &    12004\\
10.= 9 $+$ \cd{diag(ind_arith)      }  &    7.855 &      332 &   50.2 &
2811 &  167.7 &    12004\\
\em 11.= 9 $+$ \cd{diag(ind_cond)}     &\em 7.819 &\em   332 &\em 50.1&\em  2811 &\em   166.2 &\em 12004\\
\hline
12.= 11 $-$ \cd{count_empties       }  &    6.750 &      350 &   47.5 &     3248 &  166.2 &    14233\\
\hline
\end{tabular}
\end{center}

{\bf Jelmagyarázat:}

1. sima = \cd{[-count_ships_occs,-count_ones_columns,-count_empties,\\
\ \ \ \ \ \ \ \ \ label(plain),filter([off]),relation(clause),diag(reif)]}\\
11. = alapértelmezés



\subsection{Dominó}

Adott egy $(n+1)\times(n+2)$-es téglalap, amelyen egy teljes $n$-es dominókészlet
összes elemét elhelyeztük, majd a határokat eltávolítottuk. A feladat a határok
helyreállítása a számok alapján. A dominókészlet elemei az
$\{\tuple{i,j} | 0 \leq i  \leq j \leq n \}$ számpároknak felelnek meg. A kiinduló
adat tehát egy $0..n$ intervallumbeli számokból álló $(n+1)\times(n+2)$-es mátrix,
amelynek elemei azt mutatják meg, hogy az adott mezőn hány pöttyöt tartalmazó féldominó
van.

Az alábbi ábrán látható egy feladat $n=3$-ra, és annak egyetlen megoldása:

\begin{alltt}
                                  ---------------------
     1   3   0   1   2            | 1 | 3   0 | 1 | 2 |
                                  |   |-------|   |   |
     3   2   0   1   3            | 3 | 2   0 | 1 | 3 |
                                  |---------------|---|
     3   3   0   0   1            | 3   3 | 0   0 | 1 |
                                  |-------|-------|   |
     2   2   1   2   0            | 2   2 | 1   2 | 0 |
                                  ---------------------

   % Bemenő adatformátum:         % A megoldás Prolog alakja:

   [[1,  3,  0,  1,  2],          [[n,  w,  e,  n,  n],
    [3,  2,  0,  1,  3],           [s,  w,  e,  s,  s],
    [3,  3,  0,  0,  1],           [w,  e,  w,  e,  n],
    [2,  2,  1,  2,  0]]           [w,  e,  w,  e,  s]]

\end{alltt}

A megoldásban a téglalap minden mezőjéről el kell dönteni, hogy azon a mezőn egy dominó
északi (\cd{n}), déli (\cd{s}), keleti (\cd{e}) vagy nyugati (\cd{w}) fele van. A
\verb'http://www.inf.bme.hu/~szeredi/hf_00s_domino.tgz' címről letölthető állomány
tesztadatai négy csoportba oszthatóak:

\begin{itemize}
\item \cd{base} --- 16 könnyű alapfeladat $n = $ 1--25 közötti méretben.
\item \cd{easy} --- 24 középnehéz feladat, többségük $n = $ 15--25 méretben.
\item \cd{diff} --- 21 nehéz feladat 28-as, és egy 30-as méretben.
\item \cd{hard} --- egy nagyon nehéz feladat 28-as méretben.
\end{itemize}

\subsubsection{A feladat modellezése}

A torpedó feladathoz hasonlóan itt is először érdemes meggondolnunk, hogy a modellezéshez
milyen változókat használjunk, és ezeknek milyen értékkészletet válasszunk. A korlátváltozók
bevezetésére fennálló lehetőségek:

\begin{itemize}
\item[a.] Minden mezőhöz egy ún.\ \emph{irányváltozó}t rendelünk, amely a
    lefedő féldominó irányát jelzi (ez az, ami a megoldásban is szerepel).
    Ezzel a megoldással az a baj, hogy körülményes a dominók egyszeri
    felhasználását biztosítani.
\item[b.] Minden dominóhoz egy ún. \emph{dominóváltozó}t rendelünk,
    amelynek értéke megmondja, hová kerül az adott dominó. Ezzel a megoldással
    viszont az a probléma, hogy körülményes a dominók át nem fedését biztosítani. 
\item[{\bf c.}] Mezőkhöz és dominókhoz is rendelünk változókat (a.+b.). Ez az
    \emph{egyik} választott megoldás.
\item[{\bf d.}] A mezők közötti választóvonalakhoz rendelünk egy 0-1 értékű ún.\
\emph{határváltozó}t, amely azt mutatja meg, hogy az adott választóvonalon egy
dominó közepe van-e, vagy pedig két dominó érintkezik rajta. Ez a \emph{másik}
    választott megoldás.
\end{itemize}

Az irányváltozók értékkészletét legegyszerűbb az \cd{n}, \cd{s}, \cd{e}, \cd{w}
konstansok valamilyen numerikus kódolása szerint megválasztani. A dominóváltozók
értékkészletét lehetne egy $\langle${\em sor,oszlop,lehelyezési\_irány}$\rangle$
hármassal modellezni, de egyszerűbb megszámozni egy adott dominó $l$ lehetséges
lehelyezési módját, és az $1..l$ számozást használni. Például a fenti elrendezésben
a 0/2-es dominó csak három különböző módon rakható le: $\langle$2, 2, vízsz$\rangle$,
$\langle$3, 4, függ$\rangle$ és $\langle$4, 4, vízsz$\rangle$. Így a dominónak
megfeleltetett változót az 1..3 értéktartományra szoríthatjuk be. A határváltozók
1 értékének ,,természetes'' jelentése lehetne, hogy az adott határvonalat az ábrán
be kell húzni. Érdemes azonban ennek negáltjával dolgozni: legyen 1 az érték akkor,
ha az adott vonal egy dominó középvonala. Ez azért jó, mert ettől az összes korlát
\cd{A+B+... \#= 1} alakú lesz.

\subsubsection{Egy lehetséges megoldás}

\enumhead{Változók, korlátok}
\begin{itemize}
\item Minden mezőhöz egy irányváltozó (\cd{I$yx$ in 1..4 $\equiv$ \{{\em n,w,s,e}\}}),
minden dominóhoz egy dominóváltozó (\cd{D$ij, 0 \leq i \leq j \leq n$}) tartozik.
\item Szomszédsági korlát: két szomszédos irányváltozó kapcsolatát adja meg, pl.\
\cd{I14\#={\em n} \#<=> I24\#={\em s}, I14\#={\em w} \#<=> I15\#={\em e}}, stb.
Egyszerűen az olyan jellegű feltételeket adja meg, mint pl. ,,ha egy irányváltozó értékét
mondjuk \cd{n}-nek (dominó északi fele) választjuk, akkor az alatta lévő mező csak
\cd{s} (dominó déli fele) lehet''.
\item Dominó-korlát: egy dominó-elhelyezésben a dominóváltozó és a lerakás
bal vagy felső mezőjének irányváltozója közötti kapcsolat. A korábbi példában pl.\  
\cd{D02\#=1 \#<=> I22\#={\em w}, D02\#=2 \#<=> I34\#={\em n}, D02\#=3 \#<=> I44\#={\em w}}.
\end{itemize}

Mivel mindegyik megvalósítandó korlát az ,,akkor és csak akkor'' logikai kapcsolatra
épül, ezért az egyetlen fontos feladatunk, hogy ezt a logikai kapcsolatot hogyan lehet
a legoptimálisabban megvalósítani. A mintamegoldás három verziót valósít meg a
\cd{csakkor_egyenlo(X,C,Y,D)} $\equiv$ \cd{X \#= C \#<=> Y \#= D} korlátra:

\begin{itemize}
        \item \cd{reif}: reifikációval (\cd{X\#=C\#<=>Y\#=D}) 
        \item \cd{ind1}: az \cd{'x=c=>y=d'} FD predikátum kétszeri hívásával,
        \item \cd{ind2}: az \cd{'x=c<=>y=d'} FD predikátum hívásával.
\end{itemize}

Ezek közül az opciólista \cd{csakkor=Cs} paraméterével választhatunk, ahol \cd{Cs} helyére
kell írni a megfelelő variáns nevét. Az \cd{ind1}, illetve \cd{ind2} variánshoz használt
FD predikátumok:

\begin{verbatim}
'x=c=>y=d'(X, C, Y, D) +:
        X in (dom(Y) /\ {D}) ? (inf..sup) \/ \({C}),
        Y in ({X} /\  \({C})) ? (inf..sup) \/ {D}.

'x=c<=>y=d'(X, C, Y, D) +:
        X in ((dom(Y) /\ {D}) ? (inf..sup) \/ \({C})) /\
             ((dom(Y) /\ \({D})) ? (inf..sup) \/ {C}),
        Y in ((dom(X) /\ {C}) ? (inf..sup) \/ \({D})) /\
             ((dom(X) /\ \({C})) ? (inf..sup) \/ {D}).
\end{verbatim}

A címkézésnél két, lényegében különböző lehetőségünk van: címkézhetünk az irányváltozók és
a dominóváltozók szerint. Ezeken belül még variálhatunk a \cd{labeling/2} paraméterezésével
is. Az opciólistában a \cd{valt=V} opció szolgál az irányváltozók és a dominóváltozók
közti váltásra (\cd{V=irany} az irányváltozók címkézése, \cd{V=domino} a dominóváltozók
címkézése), a \cd{labeling/2} járulékos paramétereit a \cd{label=LOpciok} segítségével
adhatjuk át. A borotválás finomhangolása a \cd{szur=Sz} és \cd{szurtek=L} opciók
alkalmazásával végezhető el. Ha \cd{szur $\neq$ ki}, akkor az irány-változókat borotváljuk,
sorra megpróbáljuk az \cd{L} elemeire behelyettesíteni, és ha ez meghiúsulást okoz, akkor
az adott elemet kivesszük a változó tartományából. \cd{szur} lehet: \cd{elott} (csak a
címkézés előtt szűrünk) vagy \cd{N} (minden \cd{N}. változó címkézése után szűrünk).
\cd{L} alapértelmezése \cd{[{\em w},{\em n}]}.

\subsubsection{Egy másik lehetséges megoldás}

\enumhead{Változók, korlátok}
\begin{itemize}
\item Minden mező keleti ill.\ déli határvonalához egy-egy határváltozó
        tartozik (\cd{E$yx$} ill.\ , \cd{S$yx$}). A határváltozó akkor
        és csak akkor 1, ha az adott vonal egy dominó középvonala. A
        táblázat külső határai 0 értékűek (behúzott vonalak).
\item Szomszédsági korlát: minden mező négy oldala közül pontosan egy lesz
egy dominó középvonala, tehát pl.\ a $(2,4)$ koordinátájú dominó-mező esetén
\cd{sum([S14,E23,S24,E24]), \#=, 1)}.
\item Lerakási korlát: egy dominó összes lerakási lehetőségeit tekintjük,
ezek középvonalai közül pontosan egy lesz 1, így a példabeli
$\langle0,2\rangle$ dominóra: \cd{sum([E22,S34,E44], \#=, 1)}.
\end{itemize}

Az előző változathoz hasonlóan itt is lényegében egyetlen korlát minél optimálisabb
megvalósításával kell foglalkozni. Ez a korlát egy változólistát kap paraméterül,
és azt a feltételt fejezi ki, hogy a lista összegének 1-nek kell lennie
(\cd{lista_osszege_1}). A lehetséges megvalósítások (az \cd{osszeg=Ossz} opción
keresztül választhatunk közöttük):

\begin{itemize}
        \item \cd{Ossz=ari(N)}: \cd{N}-nél nem hosszabb listákra aritmetikai korláttal,
	egyébként (\cd{N}-nél hosszabb listákra) a \cd{sum/3} korláttal
        \item \cd{Ossz=ind(N)}: \cd{N}-nél nem hosszabb listákra FD predikátummal,
	egyébként (\cd{N}-nél hosszabb listákra) a \cd{sum/3} korláttal
        \item \cd{Ossz=sum}: mindig a \cd{sum/3} korláttal
\end{itemize}

Mivel az FD predikátumok kötött számú változóval dolgoznak, ezért az \cd{Ossz=ind(N)}
esetben szét kell választanunk az egyes eseteket, valahogy így:

\begin{verbatim}
osszege1(A, B) +:              A+B #= 1.
osszege1(A, B, C) +:           A+B+C #= 1.
osszege1(A, B, C, D) +:        A+B+C+D #= 1.
(...)
\end{verbatim}

\subsubsection{Futási eredmények}

Az alábbi időeredmények az összes megoldás megtalálására vonatkoznak, és egy
DEC Alpha 433 MHz-es gépen születtek. A táblázatokban lévő adatpárok a futási
időt (mp) és a visszalépések számát jelentik. A dőlt betűs sorok a viszonyítási
alapot jelzik, a felkiáltójel azt mutatja, hogy időtúllépés is volt a tesztadatok
között. A keretezés a legjobb időt, illetve visszalépés-számot jelenti.

\begin{center}
\begin{tabular}{|l|rr|rr|rr|rr|}
\hline
Opciók/példa   & \multicolumn{2}{l|}{\cd{base}}      & \multicolumn{2}{l|}{\cd{easy}}  & \multicolumn{2}{l|}{\cd{diff}}   & \multicolumn{2}{l|}{\cd{hard}}   \\
%%%%%%%%%%%%%%%%%%%
\hline
\multicolumn{9}{|l|}{\cd{{\rm 1.\ változat},csakkor=ind1,valt=domino,label=[],szur=2,szurtek=[1,2]}}\\
\hline
{\tt\em szur=2}            &\em  5.44&\em   1&\em  26.6&\em  28&\em4001.7&\em4950&\em1162.9&\em1448\\
\cd{szur=1,label=[ff]}    &     5.87&      1&     27.6&      5&   3900.6&   1168&    554.4&    \fbox{159}\\
\cd{szur=2,label=[ff]}    &     5.48&      1&     25.8&     13&   3222.9&   2074&    446.9&    288\\
\cd{szur=3,label=[ff]}    &     5.36&      1&     25.7&     19&   3232.6&   3597&    \fbox{429.3}&    477\\
\cd{label=[ffc]}          &     5.49&      1&     23.7&      7&  !9885.8&   6403&   3902.0&   2795\\
\cd{csakkor=ind2}         &     5.14&      1&     26.4&     28&   4250.9&   4950&   1233.0&   1448\\
\cd{csakkor=reif}         &     6.87&      1&     33.5&     28&   4573.2&   4950&   1320.2&   1448\\
\cd{szurtek=[1]}          &     4.98&      9&     34.1&     92&   6375.0&  13824&   1976.5&   3566\\
\cd{szur=elott}           &     5.09&      1&     25.1&   1722&         &       &         &       \\
\cd{szur=ki}              &     38.6&   9K  &    590  &   157K&         &       &         &       \\
\hline
\multicolumn{9}{|l|}{\cd{{\rm 1.\ változat},csakkor=ind1,valt=irany,label=[],szur=2,szurtek=[1,2]}}\\
\hline
{\tt\em label=[]}          &\em  5.39&\em   1&\em  23.4&\em  10&\em2138.1&\em1377&\em3362.9&\em2326\\
\cd{label=[ff]}           &     5.40&      1&     23.4&     10&   2137.9&   1377&   3376.5&   2326\\
\cd{label=[ffc]}          &     5.42&      1&     24.1&     10& !15036.1&  10155&  !7199.7&   4380\\
\cd{szurtek=[1]}          &     4.94&      3&     29.4&     45&   3240.2&   4000&   6077.2&   7782\\
\hline
\multicolumn{9}{|l|}{\cd{{\rm 2.\ változat},osszeg=ind(5),label=[],szur=2,szurtek=[1]}}\\
\hline
{\tt\em szur=2}            &\em  2.10&\em   1&\em  \fbox{11.5}&\em   8&\em\fbox{1045.9}&\em1399&\em1607.0&\em2254\\
\cd{szur=1}               &     2.28&      1&     11.9&      \fbox{3}&   1294.7&    \fbox{787}&   1977.9&   1277\\
\cd{szur=3}               &     2.04&      1&     11.5&     20&   1051.2&   2436&   1583.1&   3851\\
\cd{osszeg=ind(4)}        &     2.18&      1&     11.9&      8&   1152.7&   1399&   1768.0&   2254\\
\cd{osszeg=ind(6)}        &     2.13&      1&     11.9&      8&   1149.2&   1399&   1765.5&   2254\\
\cd{osszeg=sum}           &     2.96&      1&     15.8&      8&   1409.3&   1399&   2263.1&   2254\\
\cd{osszeg=ari(5)}        &     2.97&      1&     15.9&      8&   1462.7&   1399&   2257.8&   2254\\
\cd{szurtek=[0]}          &     \fbox{1.86}&      2&     15.1&    103&   2104.6&  10719&   3211.3&  17300\\
\cd{szurtek=[0,1]}        &     2.00&      1&     12.3&      7&   1182.2&   1324&   1823.7&   2150\\
\cd{label=[ff]}           &     2.12&      1&     11.7&      8&   1132.3&   1399&   1735.2&   2254\\
\cd{label=[ffc]}          &     2.14&      1&     12.4&      8&   2189.5&   2841&   2672.1&   3732\\
\hline
\multicolumn{9}{|l|}{\cd{{\rm 2.\ változat},szur=ki,label=[],  \ \ \ \ {\rm rövidítések:} l => lerak  sz => szomsz}}\\
\hline
\cd{osszeg=ind(5)}        &     3.31&    818&     57.0&  21181&         &       &         &       \\
\cd{l=ind(5),sz=sum}      &     4.61&    818&     78.6&  21181&         &       &         &       \\
\cd{l=sum,sz=ind(5)}      &     3.97&    818&     62.8&  21181&         &       &         &       \\
\cd{osszeg=sum}           &     4.57&    818&     74.8&  21181&         &       &         &       \\
%%%%%%%%%%%%%%%%%%
\hline
\end{tabular}
\end{center}
