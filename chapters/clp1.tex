\chapter{Constraint Logic Programming (CLP)}

A \emph{CLP} (\emph{Constraint Logic Programming}) a logikai programozás
egy új irányzata. Alapvető tulajdonsága, hogy a program tartalmazhat
változók értékeire való megszorításokat, \emph{korlát}okat
(\emph{constraint}). Az alábbi táblázatban összefoglaljuk a legelterjedtebb
logikai programozási nyelv, a Prolog alapelemeit és ezek CLP megfelelőit.

\btab{l|l}
Prolog                 & CLP \\
\hline
hívás                  & hívás vagy constraint \\
egyesítés              & korlátmegoldás (constraint solving) \\
válasz-behelyettesítés & válasz-korlát
\etab

A továbbiakban a CLP séma és a Prolog nyelv ötvözéséből keletkezett
programozási eszközökkel fogunk foglalkozni.

\section{A CLP nyelv elemei}

Minden CLP megvalósítás egy $\mathcal{X}$ adattartományon és az ezen
értelmezett korlátokra (relációkra) vonatkozó ,,erős'' következtetési
mechanizmus. Az $\mathcal{X}$ adattartomány különféle megválasztásaiból
más-más CLP sémák adódnak. Néhány példa:

\bul
\item $\mathcal{X}$ = $\set{R}$ vagy $\set{Q}$ (racionális vagy valós számok).\\
Itt korlátoknak tekinthetjük a racionális vagy valós számok közt fennálló
lineáris egyenlőségeket egyenlőtlenségeket, következtetési mechanizmusnak
pedig a Gauss-eliminációt és a szimplex módszert.

\item $\mathcal{X}$ = FD (egész számok véges tartománya, angolul FD =
Finite Domain). \\
Korlátoknak vehetjük a különféle aritmetikai és kombinatorikai
relációkat, következtetési mechanizmusnak pedig a mesterséges intelligencia
kutatásokból ismert CSP (Constraint Solving Problem, korlátkielégítési
probléma) módszereket.

\item $\mathcal{X}$ = $\set{B}$ (logikai igaz és hamis értékek). \\
Itt a
korlátok a predikátum kalkulusban fennálló relációk, a következtetési
mechanizmus pedig szintén a mesterséges intelligencia területéhez
tartozó SAT (satisfiability, Boole-kielégíthetőség) módszer.
\eul

A függvényeket és a relációkat CLP-ben -- a Prolog-tól eltérően -- nem szintaktikusan, hanem szemantikusan kezeljük, a Prologban a
program dolga, hogy jelentést tulajdonítson egy függvény
(struktúra) - kifejezésnek (pl. az \cd{is/2} predikátum).  A Prolog
csak a szintaktikus egyesítést (a \cd{=/2} műveletet) ismeri, ami
például az \cd{X+2=Y+3} kifejezést sikertelenül próbálja
feldolgozni, így a hívás meghiúsul. Ha ezt a kifejezést egy
alaphalmaz, domain felett nézzük, pl. a valós számok fölött
(CLP($R$)), akkor azt a matematikailag helyes megoldást kapjuk, hogy
\cd{X=Y+1}. Ahhoz, hogy ezt az eredményt kaphassuk, szükség van a
\emph{korlátmegoldó}ra (\emph{constraint solver}), ami a változókból,
értékekből, függvényekből és a relációkból álló korlátokat ki tudja
értékelni, és ellenőrizni tudja a konzisztenciájukat. Ha egy új
korlát hozzávételével a tár inkonzisztenssé válna, azt a
korlátmegoldó észreveszi, és a programnak az aktuális
végrehajtási ága meghiúsul. Ilyenkor a Prolog végrehajtás szabályai szerint
visszalépés következik be.
\br
A fent elmondottak formalizálva a következőképpen néznek ki:
\br
Egy CLP rendszer egy $\tuple{\mathcal{D},\mathcal{F},\mathcal{R},\mathcal{S}}$
struktúrával írható le, ahol az egyes elemek jelentése:

\bul
\item $\mathcal{D}$ -- egy tartomány, pl. a valós számok ($\set{R}$),
a racionális számok ($\set{Q}$), az egész számok ($\set{N}$),
a Boole-értékek ($\set{B}$), karakterfüzérek, listák, Prolog végrehajtási
fák (Herbrand-fák, $\set{H}$) tartománya
\item $\mathcal{F}$ -- a fenti tartományon értelmezett függvények halmaza,
pl. $+ , - , *, \land, \lor$
\item $\mathcal{R}$ -- a fenti tartományon értelmezett relációk halmaza,
pl. $=, \neq, <, >, \in$
\item $\mathcal{S}$ -- egy korlátmegoldó algoritmus $\tuple{\mathcal{D},
\mathcal{F},\mathcal{R}}$-re, azaz a $\mathcal{D}$ tartományon értelmezett
$\mathcal{F} \cup \mathcal{R}$ halmazbeli jelekből felépített korlátokra 
\eul

\section{CLP szintaxis és deklaratív szemantika}

{\bf Szintaxis:}

\btab{l|l}
     program: &  klózok halmaza \\
\hline
     klóz: & 
     \cd{P:- G$_1$, \dots, G$_n$}, ahol \cd{G$_i$} vagy cél vagy constraint\\
\hline
     deklarat{\'\i}v olvasata: &  
     \cd{P} igaz, ha \cd{G$_1$, \dots, G$_n$} mind igaz\\
\hline
     kérdés: & \cd{?- G$_1$ , \dots, G$_n$} \\
\hline
     válasz egy kérdésre: &  
     korlátoknak egy olyan konjunkciója, \\
     & amelyből a kérdés következik\\
\etab

\section{CLP procedurális szemantika}

A CLP séma szemantikája leírható, mint egy kezdeti célból történő levezetés,
amely felhasználja a program klózait. A levezetés egy állapota egy
$\tuple{G,s}$ párral jellemezhető, ahol:

\bul
\item $G$ a megoldandó célok és korlátok konjunkciója
\item $s$ egy \emph{korlát-tár}, amely az eddig felhalmozott korlátokat
tartalmazza. A korlát-tár tartalma mindig \emph{kielégíthető}, kezdetben
pedig üres.
\eul

Sok CLP megvalósításban a korlát-tár csak a korlátok egy bizonyos osztályát
tárolhatja, ezeket a korlátokat \emph{egyszerű korlát}oknak nevezzük,
a korlát-tárba be nem tehető korlátokat pedig \emph{összetett korlát}oknak.
Például a SICStus \clpfd könyvtárának használatakor az egyszerű korlátok csak az
$\in$ relációt tartalmazhatják. Az összetett korlátok felfüggesztve,
\emph{démon}ként várnak arra, hogy a korlát-megoldónak segíthessenek.
A továbbiakban az egyszerű korlátokat kisbetűkkel (pl. $c$), az általános
jellegű korlátokat pedig nagybetűkkel (pl. $C$) jelöljük.
\br
Procedurális értelmezésben egy \cd{P :- G$_1$, G$_2$, \dots, G$_n$} klóz jelentése
a következő: \cd{P} elvégzéséhez el kell végezni \cd{G$_1$}-et, \cd{G$_2$}-t,
\dots \cd{G$_n$}-et.

\enum{A végrehajtás alaplépése:}{
\item A végrehajtandó cél egy részcéljának, \cd{P}-nek determinisztikus
kiválasztása
\item Egy \cd{P}-re illeszkedő klóz, \cd{P'} nemdeterminisztikus kiválasztása
\item \cd{P'} végrehajtása és a korlát-tár konzisztenciájának ellenőrzése
\item Ha a korlát-tár konzisztens, akkor az eljárás folytatható a következő
részcéllal, ha viszont nem konzisztens, akkor a végrehajtás ezen ága
meghiúsul, visszalépés következik be, melynek során a hagyományos visszalépési
mechanizmus mellett a korlát-tár tartalmát is vissza kell fejteni a legutóbbi
választási pontig.
}
A végrehajtás minden $\tuple{G,s}$ állapotában teljesül, hogy $s$ konzisztens,
és $G \land s \Rightarrow Q$, ahol $Q$ a kezdő kérdés. A végrehajtás akkor áll meg, ha
egy olyan $\tuple{G_c,s_c}$ állapotba kerültünk, ahol $G_c$-re már egyetlen
következtetési lépés sem végezhető el. Ekkor a végrehajtás eredménye az
$s_c$ korlát-tár (vagy annak egy adott változóhalmazra való vetítése
a többi változó egzisztenciális kvantifikálásával), valamint az esetlegesen
fennmaradó $G_c$ korlátok.

\label{erosit}

\enum{Általános következtetési lépések}{
\item \emph{rezolúció}:\\
   \tb\cd{P} \& \cd{G}, $s$\te\ $\Rightarrow$
    \tb\cd{\cd{G}$_1$} \& \dots\ \& \cd{\cd{G}$_n$} \& \cd{G}, \cd{P = P$'$} $\land$ $s$\te,
    feltéve, hogy a programban van egy \cd{P$'$ :- G$_1$}, \dots, \cd{G$_n$} klóz.

\item \emph{korlát-megoldás}:\\

    \tb \cd{c} \& \cd{G}, $s$\te\ $\Rightarrow$ \tb \cd{G}, $s$ $\land$ \cd{c}\te,
    tehát egy egyszerű korlát bekerülhet a korlát-tárba, feltéve, ha
    a korlát-tár továbbra is konzisztens marad
        
\item \emph{korlát-erősítés}:\\

    \tb \cd{C} \& \cd{G}, $s$\te\ $\Rightarrow$
    \tb \cd{C}$'$ \& \cd{G}, $s \land$ \cd{c}\te
    ha $s$-ből következik, hogy \cd{C} ekvivalens (\cd{C}$'$ $\land$ \cd{c})-vel
    (\cd{C}$'$ = \cd{C} is lehet) és $s$ $\land$ \cd{c} konzisztens marad.
    Tehát egy összetett korlát \emph{erősíti} a korlát-tár tartalmát, ha a
    korlát ekvivalens egy egyszerű korlát és egy másik összetett korlát
    konjunkciójával. Ilyenkor az egyszerű korlát bekerül a korlát-tárba,
    az új összetett korlát pedig visszakerül a célsorozatba.
}

A korlát-erősítésnek két speciális esetét érdemes megjegyezni:

\begin{enumerate}
\item \cd{C}=\cd{C'}. Ilyenkor a \cd{c}-re vonatkozó feltétel az, hogy
$s\Rightarrow($\cd{C}$\Rightarrow$\cd{c}$)$, azaz a tárból és a \cd{C} korlátból
megpróbálunk egy egyszerű \cd{c} korlátot kikövetkeztetni.
\item \cd{C'=true}, azaz az $s$ korlát-tár mellett \cd{C}-nek létezik egy
vele ekvivalens \cd{c} párja, ahol \cd{c} már egyszerű, így felvehetjük
a korlát-tárba, \cd{C'}-t pedig eldobhatjuk.
\end{enumerate}

Egy példa korlát-erősítésre: a korlát-tár tartalma legyen ${Y>3}$, az
összetett korlát pedig $X>Y*Y$. Ekkor $X>Y*Y \land Y>3 \Rightarrow X>9$, ami már
felvehető a korlát-tárba. Az $X>Y*Y$ korlátnak továbbra is ,,démonként'' életben
kell maradnia, hogy amikor a későbbiekben $Y$ tartományára további szűkítéseket
teszünk, akkor az egyúttal módosítani tudja $X$ tartományát is.

\enum{A korlátmegoldó algoritmussal szemben támasztott követelmények}{
\item \emph{teljesség}: egyszerű korlátok konjunkciójáról mindig el tudja
dönteni, hogy az konzisztens-e
\item \emph{inkrementalitás}: új korlát felvételekor ne bizonyítsa újra a
teljes tár konzisztenciáját, csak azokat a korlátokat, amelyeket az új
korlát felvétele érint
\item \emph{visszalépés támogatása}: a levezetés során ellentmondás esetén
vissza tudja csinálni a korlátok felvételét
\item \emph{hatékonyság}
}

\section{A CLP rendszerek felhasználási lehetőségei}

\begin{itemize}
\item {\bf Ipari erőforrás optimalizálás}: termék- és gépkonfiguráció,
gyártásütemezés, emberi erőforrások ütemezése, logisztikai tervezés

\item {\bf Közlekedés, szállítás}: repülőtéri allokációs feladatok
(beszállókapu, poggyász-szalag stb.), repülő-személyzet járatokhoz rendelése
menetrendkészítés, forgalomtervezés

\item {\bf Távközlés, elektronika}: GSM átjátszók frekvencia-kiosztása
lokális mobiltelefon-hálózat tervezése, áramkörtervezés és verifikálás

\item {\bf Egyéb}: szabászati alkalmazások, grafikus megjelenítés megtervezése,
multimédia szinkronizáció, légifelvételek elemzése
\end{itemize}
