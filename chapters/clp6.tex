\clearpage

\chapter{Az \fdbg nyomkövető csomag}

\label{fdbg}
Ebben a fejezetben a \clpfd programok nyomkövetésére szolgáló \fdbg csomagot
mutatjuk be részletesebben. Az \fdbg könyvtár megalkotásakor a szerzők
(\emph{Szeredi Tamás} és \emph{Hanák Dávid}) az alábbi szempontokat tartották
szem előtt:

\begin{itemize}
\item követhető legyen a véges tartományú (röviden: FD) korlát változók
  tartományainak szűkülése;
\item a programozó értesüljön a korlátok felébredéséről, kilépéséről és
  hatásairól, valamint az egyes címkézési lépésekről és hatásukról;
\item jól olvasható formában lehessen kiírni FD változókat tartalmazó
  kifejezéseket.
\end{itemize}

\section{Alapfogalmak}

A csomag használatához szükséges megismerkednünk néhány alapvető fogalommal:
\br
\definicio {\tt \em clpfd} {\em esemény}nek nevezzük egy globális korlát felébredését
vagy bármely címkézési eseményt (címkézés kezdete, címkézési lépés vagy címkézés
meghiúsulása).
\br
\definicio \emph{Megjelenítő}nek (\emph{visualizer}) nevezzük a \clpfd eseményekre
reagáló predikátumot. Általában az a feladata, hogy az eseményt valamilyen formában
kiírja. A tényleges esemény bekövetkezése és így az esemény hatásainak érvényesülése
\emph{előtt} hívódik meg. Mindkét fajta \clpfd eseményhez külön megjelenítő tartozik,
így beszélhetünk \emph{korlát-megjelenítő}ről és \emph{címkézés-megjelenítő}ről.
\br
\definicio \emph{Jelmagyarázat}nak (\emph{legend}) nevezzük az éppen megfigyelt korlát
után kiíródó szövegrészt, amely rendszerint a korlátban részt vevő változókat, a
hozzájuk kapcsolódó tartományokat és a vizsgált korlát végrehajtásával kapcsolatos
következtetéseket tartalmazza.
\br
Az \fdbg által csak a globális korlátok követhetőek nyomon, az indexikálisok nem,
viszont a globális korlátok közül a felhasználói és a beépített korlátok ugyanolyan
módon kezelhetőek. Az \fdbg nyomkövetés bekapcsolt állapota esetén a formula-korlátokból
mindenképp globális korlátok képződnek, nem pedig indexikálisok.
\br
Az \fdbg könyvtár külön segédeljárásokat biztosít a kifejezésekben található FD
változók megjelöléséhez (\emph{annotálás}), az annotált kifejezések jól olvasható
megjelenítéséhez, valamint jelmagyarázat előkészítéséhez és kiírásához.
\br
Az \fdbg könyvtár szolgáltatásait az alábbi paranccsal lehet igénybe venni:

\begin{prologcode}
| ?- use_module(library(fdbg)).
\end{prologcode}

\section{A nyomkövetés be- és kikapcsolása}

\begin{itemize}
\item \bcd{fdbg\_on}\\ \bcd{fdbg\_on(+{\em Options})}\\
Engedélyezi a nyomkövetést alapértelmezett vagy megadott beállításokkal. A
nyomkövetést az \cd{fdbg\_output} álnevű (stream alias) folyamra írja a
rendszer, alaphelyzetben ez a pillanatnyi kimeneti folyam ({\em current
output stream}) lesz. Legfontosabb opciók:\\
\begin{itemize}
\item \cd{file({\em Filename}, {\em Mode})}\\
  A megjelenítők kimenete a \cd{\em Filename} nevű állományba irányítódik
  át, amely az \cd{fdbg\_on/1} hívásakor nyílik meg \cd{\em Mode} módban.
  \cd{Mode} lehet \cd{write} vagy \cd{append}.
\item \cd{stream({\em Stream})}\\
  A megjelenítők kimenete a \cd{\em Stream} folyamra irányítódik át.
\item \cd{constraint\_hook({\em Goal})}\\
  \cd{\em Goal} két argumentummal kiegészítve meghívódik a korlátok
  felébredésekor. Alapértelmezésben ez az \cd{fdbg\_show/2} eljárás, ld.~később.
  Ezzel a paraméterrel lehet felüldefiniálni a korlát-megjelenítőt.
\item \cd{labeling\_hook({\em Goal})}\\
  \cd{\em Goal} három argumentummal kiegészítve meghívódik minden címkézési
  eseménykor. Alapértelmezésben ez az \cd{fdbg\_label\_show/3}, ld.~később.
  Ezzel a paraméterrel lehet felüldefiniálni a címkézés-megjelenítőt.
\item \cd{no_constraint_hook}, \cd{no_labeling_hook}\\ 
  Nem lesz adott fajtájú megjelenítő.
\end{itemize}

\item \bcd{fdbg\_off}\\
  Kikapcsolja a nyomkövetést, lezárja a \cd{file} opció hatására megnyitott
  állományt (ha volt ilyen).
\end{itemize}

Kimenet átirányítása, beépített megjelenítő, nincs címkézési nyomkövetés:

\begin{prologcode}
| ?- fdbg_on([file('my_log.txt', append), no_labeling_hook]).
\end{prologcode}

Kimenet átirányítása szabványos folyamra, saját és beépített
megjelenítő együttes használata:

\begin{prologcode}
| ?- fdbg_on([constraint_hook(fdbg_show), constraint_hook(my_show),
              stream(user_error)]).
\end{prologcode}

\section{Kifejezések elnevezése}

Az \fdbg által produkált kimeneten az FD változók alaphelyzetben \cd{<fdvar_1>},
\cd{<fdvar_2>}, \ldots formában jelennek meg, ami megnehezíti az olvashatóságot.
Éppen ezért minden FD változóhoz egy saját nevet rendelhetünk, ami a kimeneten
az \cd{<fdvar_{\em x}>} forma helyett fog megjelenni. Lehetőség van arra is,
hogy egy egész kifejezéshez rendeljünk nevet. Egy kifejezés elnevezésekor a
kifejezésben szereplő összes változóhoz egy-egy származtatott név
rendelődik -- ez a név a megadott névből és a változó kiválasztójából
keletkezik (struktúra argumentum-sorszámok ill.\ lista indexek sorozata). A
létrehozott nevek egy globális listába kerülnek, amely mindig egyetlen toplevel
híváshoz tartozik, nem vivődik át a következő toplevel hívásra (\emph{illékony}).

\enum{Predikátumok}{
\item \bcd{fdbg\_assign_name(+{\em Name}, +{\em Term})} \\
    A \cd{\em Term} kifejezéshez a \cd{\em Name} nevet rendeli az aktuális
    toplevel hívásban.
\item \bcd{fdbg\_current\_name(?{\em Name}, -{\em Term})} \\
    Lekérdez egy kifejezést (változót) a globális listából a neve alapjén.
    Ha \cd{\em Name} változó, akkor felsorolja az összes tárolt név-kifejezés párt.
\item \bcd{fdbg\_get\_name(+{\em Term}, -\em{Name})} \\
    \cd{\em Name} a \cd{\em Term} kifejezéshez rendelt név.  Ha \cd{\em Term}-nek még nincs
    neve, automatikusan hozzárendelődik egy.
}

Pl.~\cd{fdbg\_assign_name(foo, bar(A, [B, C]))} hatására a következő nevek
generálódnak:

\begin{center}
\begin{tabular}{|l|l|l|}
\hline
\multicolumn{1}{|c|}{név} &
\multicolumn{1}{c|}{kifejezés} &
\multicolumn{1}{c|}{megjegyzés} \\
\hline
\tt foo       & \tt bar(A, [B, C]) & a teljes kifejezés \\
\tt foo\_1    & \tt A & \cd{bar} első argumentuma \\
\tt foo\_2\_1 & \tt B & \cd{bar} második argumentumának első eleme \\
\tt foo\_2\_2 & \tt C & \cd{bar} második argumentumának második eleme \\
\hline
\end{tabular}
\end{center}

\section{Egyszerűbb \fdbg nyomkövetési példák}

\begin{prologcode}
| ?- use_module([library(clpfd),library(fdbg)]).

| ?- fdbg_on.
% The clp(fd) debugger is switched on
% advice
| ?- Xs=[X1,X2], fdbg_assign_name(Xs, 'X'),
     domain(Xs, 1, 6), X1+X2 #= 8, X2 #>= 2*X1+1.
\end{prologcode}
\begin{prologcode}
domain([<X_1>,<X_2>],1,6)           X_1 = inf..sup -> 1..6
                                    X_2 = inf..sup -> 1..6
                                    Constraint exited.    

<X_1>+<X_2>#=8                      X_1 = 1..6 -> 2..6
                                    X_2 = 1..6 -> 2..6  

<X_2>#>=2*<X_1>+1                   X_2 = 2..6 -> 5..6
                                    X_1 = 2..6 -> {2} 
                                    Constraint exited.

<X_2>#=6    [2+<X_2>#=8 (*)]        X_2 = 5..6 -> {6} 
                                    Constraint exited.

X1 = 2, X2 = 6 ?
% advice
\end{prologcode}

A \cd{(*)} olvashatóbb alak a \cd{library(fdbg)} négy sorának kikommentezésével
állitható elő.

\begin{prologcode}
| ?- X in 1..4, labeling([bisect], [X]).

<fdvar_1> in 1..4                   fdvar_1 = inf..sup -> 1..4
                                    Constraint exited.        

Labeling [2, <fdvar_1>]: starting in range 1..4.
Labeling [2, <fdvar_1>]: bisect: <fdvar_1> =< 2
        Labeling [4, <fdvar_1>]: starting in range 1..2.
        Labeling [4, <fdvar_1>]: bisect: <fdvar_1> =< 1     
X = 1 ? ;
        Labeling [4, <fdvar_1>]: bisect: <fdvar_1> >= 2
X = 2 ? ;
        Labeling [4, <fdvar_1>]: failed.
Labeling [2, <fdvar_1>]: bisect: <fdvar_1> >= 3
        Labeling [8, <fdvar_1>]: starting in range 3..4.
        Labeling [8, <fdvar_1>]: bisect: <fdvar_1> =< 3
X = 3 ? ;
        Labeling [8, <fdvar_1>]: bisect: <fdvar_1> >= 4
X = 4 ? ;
        Labeling [8, <fdvar_1>]: failed.
Labeling [2, <fdvar_1>]: failed.
no
\end{prologcode}

\section{Beépített megjelenítők}

Az \fdbg könyvtár egy-egy alapértelmezett megjelenítőt bocsájt a felhasználó
rendelkezésére. A korlát-megjelenítőt az \cd{fdbg_show/2}, a címkézés-megjelenítőt
az \cd{fdbg_label_show/3} predikátum tartalmazza. Ha az \cd{fdbg_on} predikátumnak
nem adunk meg külön korlát-, illetve címkézés-megjelenítőt, akkor ezeket a megjelenítőket
alkalmazza.

\begin{itemize}
\item \bcd{fdbg\_show(+{\em Constraint}, +{\em Actions})} \\
Beépített korlát-megjelenítő. A \cd{dispatch_global}-ból való kilépéskor hívódik
meg. Megkapja az aktuális korlátot és az általa előállított akciólistát, majd
ennek alapján megjeleníti a korlátot és a hozzá tartozó jelmagyarázatot. 

,,Szimulált'' példa-hívás \cd{X1=3} és \cd{X2=3} szűkítésekkel:

\begin{prologcode}
| ?- Xs=[X1,X2,X3], fdbg_assign_name(Xs, 'X'), 
     domain(Xs, 1, 3), X3 #\= 3, fdbg_on, 
     fdbg_show(exactly(3,Xs,2),[exit,X1=3,X2=3]).

exactly(3,[<X_1>,<X_2>,<X_3>],2)
    X_1 = 1..3 -> {3}
    X_2 = 1..3 -> {3}
    X_3 = 1..2
    Constraint exited.
\end{prologcode}

\item \bcd{fdbg\_label\_show(+{\em Event}, +\em{ID}, +\em{Variable})}\\
Beépített címkézés-megjelenítő. Címkézési eseménykor (kezdet, szűkítés,
meghiúsulás) hívódik meg. \cd{\em Event}-ben megkapja a hívást kiváltó
eseményt, \cd{\em ID}-ben a címkézési lépés azonosítóját, \cd{\em Variable}-ben
pedig magát a címkézendő változót. \cd{\em Event} megegyezik a
\cd{start} atommal, ha a kiváltó esemény a címkézés kezdete, megegyezik a \cd{fail}
atommal, ha a kiváltó esemény a címkézés meghiúsulása. Példa:

\begin{prologcode}
| ?- fdbg_assign_name(X, 'X'), X in {1,3}, fdbg_on,
     indomain(X).
% The clp(fd) debugger is switched on
Labeling [1, <X>]: starting in range {1}\/{3}.
Labeling [1, <X>]: indomain_up: <X> = 1

X = 1 ? ;
Labeling [1, <X>]: indomain_up: <X> = 3

X = 3 ? ;
Labeling [1, <X>]: failed.

no
\end{prologcode}

\end{itemize}

A fenti kimenet elkészítése során végrehajtott megjelenítő-hívások:

\begin{prologcode}
fdbg_label_show(start,1,X)
fdbg_label_show(step('$labeling_step'(X,=,1,indomain_up)),1,X)
fdbg_label_show(step('$labeling_step'(X,=,3,indomain_up)),1,X)
fdbg_label_show(fail,1,X)
\end{prologcode}

\section{Testreszabás kampó-eljárásokkal}

Az alábbi kampó-eljárások a következő három argumentummal rendelkeznek:

\begin{itemize}
\item \emph{Name} --- az FD változó neve
\item \emph{Variable} --- maga a változó
\item \emph{FDSetAfter} --- a változó tartománya, \emph{miután} az aktuális
  korlát elvégezte rajta a szűkítéseket
\end{itemize}

A kampó-eljárások felülbírálásához azokat \emph{multifile} eljárásként kell definiálni.

\begin{itemize}
\item \bcd{fdbg:fdvar\_portray(+{\em Name}, +{\em Variable}, +{\em FDSetAfter})}\\
  A kiírt korlátokban szereplő változók megjelenésének megváltoztatására
  szolgál. Az alapértelmezett viselkedés \textsl{Name} kiírása kacsacsőrök
  között. Az alábbi példa a változó neve mellett a változó régi tartományát is
  kiírja:

\begin{prologcode}
:- multifile fdbg:fdvar_portray/3.

fdbg:fdvar_portray(Name, Var, _) :-
       fd_set(Var, Set), fdset_to_range(Set, Range),
       format('<~p = ~p>', [Name,Range]).
\end{prologcode}

\item \bcd{fdbg:legend\_portray(+{\em Name}, +{\em Variable}, +{\em FDSetAfter})}\\
  A jelmagyarázat minden sorára meghívódik, és így végigfut a jelmagyarázatban
  szereplő változókon. A sorokat alapértelmezésben négy szóköz nyitja és egy
  újsor karakter zárja, ezeket nem kell kiírni, de nem is lehet őket felülbírálni.
  Az alábbi példa a változó tartományának szűkülését lista formában tünteti fel:

\begin{prologcode}
:- multifile fdbg:legend_portray/3.

fdbg:legend_portray(Name, Var, Set) :-
        fd_set(Var, Set0), fdset_to_list(Set0, L0),
        (   Set0 == Set
        ->  format("~p = ~p", [Name, L0])
        ;   fdset_to_list(Set, L),
            format("~p = ~p -> ~p", [Name,L0,L])
        ).
\end{prologcode}
\end{itemize}

A példák kimenete összevetve az alapértelmezettel:

\begin{center}\tt\begin{tabular}{l|l}
\multicolumn{1}{c}{\rm \em Eredeti alak} &
\multicolumn{1}{c}{\rm \em Testreszabott alak} \\

exactly(3,[<X>,2],1)~~~~~~~ & exactly(3,[<X = 1..3>,2],1) \\
    X = 1..3 -> \{3\}       &     X = [1,2,3] -> [3]      \\
    Constraint exited.      &     Constraint exited. \\
\end{tabular}\end{center}

\section{Testreszabás saját megjelenítővel}

\begin{itemize}
\item \bcd{{\em my\_global\_visualizer}(+{\em Arg1}, +{\em Arg2}, \ldots, {\em Constraint},
{\em Actions})}\\
\cd{\em my\_global\_visualizer} helyére tetszőleges predikátumnév kerülhet, \cd{\em Arg1},
\cd{\em Arg2} \ldots a predikátum által használt saját argumentumok, \cd{\em Constraint}
és \cd{\em Actions} pedig az a két paraméter, amit az \cd{fdbg\_show/2} rak a paraméterlista
végére. \cd{\em Constraint} az éppen felébredt korlátot, \cd{\em Actions} pedig a korlát
által visszaadott akciólistát tartalmazza. A korlát-megjelenítő nevét és paraméterlistáját
(az utolsó kettő kivételével) az \cd{fdbg_on} opciólistájában kell átadni:\\
\cd{fdbg\_on(constraint\_hook({\em my\_global\_visualizer}({\em Arg1}, {\em Arg2},
\ldots)))}.

\item \bcd{{\em my\_labeling\_visualizer}(+{\em Arg1}, +{\em Arg2}, \ldots, {\em Event},
{\em ID}, {\em Var})}\\
\cd{\em Event} a megjelenítő meghívását kiváltó címkézési esemény, a következőek szerint:
\begin{itemize}
\item \cd{start} --- címkézés kezdete
\item \cd{fail} --- címkézés meghiúsulása
\item \cd{step({\em Step})} --- címkézési lépés, amelyet \cd{\em Step} ír le
\end{itemize}
\cd{\em ID} a címkézés azonosítója, \cd{\em Var} pedig a címkézett változó. A
címkézés-megjelenítő nevét és paraméterlistáját (az utolsó kettő kivételével) az
\cd{fdbg_on} opciólistájában kell átadni:\\
\cd{fdbg\_on(labeling\_hook({\em my\_labeling\_visualizer}({\em Arg1}, {\em Arg2},
\ldots)))}.
\end{itemize}

Érdemes egy pillantást vetni az \cd{fdbg\_show/2} beépített megjelenítő kódjára is,
mert ez irányelvet adhat saját megjelenítő írásához:

\begin{prologcode}
fdbg_show(Constraint, Actions) :-
        fdbg_annotate(Constraint, Actions, AnnotC, CVars),
        print(fdbg_output, AnnotC),
        nl(fdbg_output),
        fdbg_legend(CVars, Actions),
        nl(fdbg_output).
\end{prologcode}

Gyakran szükség lehet arra, hogy csak bizonyos korlátok működését vizsgáljuk,
ilyenkor jön jól egy saját \cd{fdbg\_show/2} eljárás:

\begin{prologcode}
filtered_show(Constraint, Actions) :-
        Constraint = scalar_product(_,_,_,_),
        fdbg_show(Constraint, Actions).
\end{prologcode}

A fenti predikátum meghiúsulhat, ha \cd{Constraint} nem \cd{scalar\_product/4} korlát,
de a megjelenítők esetében a meghiúsulás nem okozza automatikusan az egész Prolog
végrehajtás meghiúsulását, ezért az ilyen jellegű megoldás megengedett. A
megjelenítő használata:

\begin{prologcode}
:- fdbg_on([constraint_hook(filtered_show), file('fdbg.log', write)]).
\end{prologcode}

\section{Egyéb segéd-predikátumok}

A változók tartományának kiírásához és az úgynevezett ~\emph{annotáláshoz} több
predikátum adott.  Ezeket használják a beépített nyomkövetők, de hívhatók
kívülről is. Ebben az alfejezetben ezeket a predikátumokat ismertetjük.

\begin{itemize}
\item \bcd{fdbg\_annotate(+{\em Term0}, -{\em Term}, -{\em Vars})} \\
  \bcd{fdbg\_annotate(+{\em Term0}, +{\em Actions}, -{\em Term}, -{\em Vars})} \\
  A \cd{\em Term0} kifejezésben található összes FD változót megjelöli,
  azaz lecseréli egy \cd{fdvar/3} struktúrára.  Ennek tartalma:
  \begin{itemize}
  \item a változó neve
  \item a változó maga (tartománya még a szűkítés előtti állapotokat
    tükrözi)
  \item egy FD halmaz, amely a változó tartománya \emph{lesz} az
    \cd{\em Actions} akciólista szűkítései után.
  \end{itemize}
  Az így kapott kifejezés lesz \cd{\em Term}, a beszúrt \cd{fdvar/3}
  struktúrák listája pedig \cd{\em Vars}.

Példa az \cd{fdbg\_annotate/3} használatára:

\begin{prologcode}
| ?- length(L, 2), domain(L, 0, 10), fdbg_assign_name(L, x), 
     L=[X1,X2], fdbg_annotate(lseq(X1,X2), Goal, _), 
     format('write(Goal) --> ~w~n', [Goal]),
     format('print(Goal) --> ~p~n', [Goal]).

write(Goal) --> lseq(fdvar(x_1,_2,[[0|10]]),fdvar(x_2,_2,[[0|10]]))
print(Goal) --> lseq(<x_1>,<x_2>)
\end{prologcode}

Látható, hogy az \cd{fdvar/3} struktúrákra az \fdbg modul definiál egy saját
\cd{portray/1} megjelenítő eljárást, ezért ha a \cd{print/1} használatával
íratjuk ki az ilyen struktúrákat, akkor azok a fenti tömör módon jelennek meg.

\item \bcd{fdbg\_legend(+{\em Vars})} \\
   \bcd{fdbg\_legend(+{\em Vars}, +{\em Actions})} \\
  Az \cd{fdbg\_annotate/3,4} használatával előálló \cd{\em Vars} változólistát
  és az \cd{\em Actions} akció-listából levonható következtetéseket jelmagyarázatként
  kiírja a következő szabályok szem előtt tartásával:
  \begin{itemize}
  \item egy sorba egy változó leírása kerül
  \item minden sor elején a változó neve szerepel
  \item a nevet a változó tartománya követi (régi \cd{->} új)
  \end{itemize}

\end{itemize}

\section{A mágikus sorozatok feladat nyomkövetése}

Ebben az alfejezetben a mágikus sorozatok feladat egy lehetséges megoldásán
mutatjuk be az \fdbg nyomkövető csomag kimenetét. A programkódban csak a 
nyomkövetés megértéséhez szükséges predikátumokat tüntettük fel:

\begin{prologcode}
magic(N, L) :-
        length(L, N),
        fdbg_assign_name(L, x), % <--- !!!
        N1 is N-1, domain(L, 0, N1),
        occurrences(L, 0, L),
%       sum(L, #=, N),
%       findall(I, between(0, N1, I), C),
%       scalar_product(C, L, #=, N),
        labeling([ff], L).

occurrences([], _, _).
occurrences([E|Ek], I, List) :-
        exactly(I, List, E), J is I+1,
        occurrences(Ek, J, List).

| ?- fdbg_on, magic(4, L).
\end{prologcode}

A kimenet vége, az utolsó címkézési lépés után:

\begin{prologcode}
exactly(0,[1,2,<x_3>,<x_4>],1)          x_3 = 0..3
                                        x_4 = 0..3
\end{prologcode}
\begin{prologcode}
exactly(2,[1,2,<x_3>,<x_4>],<x_3>)      x_3 = 0..3 -> 1..3
                                        x_4 = 0..3        
\end{prologcode}
\begin{prologcode}
exactly(3,[1,2,<x_3>,<x_4>],<x_4>)      x_3 = 1..3        
                                        x_4 = 0..3 -> 0..2
\end{prologcode}
\begin{prologcode}
exactly(1,[1,2,<x_3>,<x_4>],2)          x_3 = 1..3
                                        x_4 = 0..2
\end{prologcode}
\begin{prologcode}
exactly(2,[1,2,<x_3>,<x_4>],<x_3>)      x_3 = 1..3
                                        x_4 = 0..2
\end{prologcode}
\begin{prologcode}
exactly(0,[1,2,<x_3>,<x_4>],1)          x_3 = 1..3        
                                        x_4 = 0..2 -> {0} 
                                        Constraint exited.
\end{prologcode}
\begin{prologcode}
exactly(1,[1,2,<x_3>,0],2)              x_3 = 1..3 -> {1} 
                                        Constraint exited.
\end{prologcode}
\begin{prologcode}
exactly(2,[1,2,1,0],1)                  Constraint exited.
\end{prologcode}
\begin{prologcode}
exactly(3,[1,2,1,0],0)                  Constraint exited.

L = [1,2,1,0] ? 
\end{prologcode}
