\clearpage

\chapter{CHR---Constraint Handling Rules\kieg}

A CHR (\emph{Constraint Handling Rules}) egy deklaratív nyelv-kiterjesztés,
amely determinisztikus kifejezés-átíráson alapul. Prolog, CLP, Haskell
vagy Java \emph{gazda}nyelvbe ágyazódva képes működni. Általános szimbolikus
(nem numerikus) felhasználói korlátok írására alkalmas, de nem tartalmaz
konzisztencia-vizsgálatot, erről a megfelelő szabályok megírásával
nekünk kell gondoskodnunk. Fő szerzője Thom Frühwirth (ECRC, LMU München, Ulm Uni.).
\br
A nyelv-kiterjesztés honlapja:
\verb'http://www.pst.informatik.uni-muenchen.de/~fruehwir/chr-intro.html'.
\br
SICStus Prologban a CHR kiterjesztést a következő paranccsal vehetjük használatba:
\begin{prologcode}
:- use_module(library(chr)).
\end{prologcode}

\section{CHR szabályok}

A CHR kiterjesztésben a korlát-tár alakulását saját magunk által írt szabályok
segítségével írhatjuk le. A szabályoknak három fajtája van:

\begin{itemize}
\item \emph{Egyszerűsítés} (\emph{simplification}):\\
 $H_1$, \ldots, $H_i$ \cd{<=>} $G_1$, \ldots, $G_j$ \cd{|} $B_1$,
 \ldots, $B_k$.
\item \emph{Propagáció} (\emph{propagation}):\\
 $H_1$, \ldots, $H_i$ \cd{==>} $G_1$, \ldots, $G_j$ \cd{|} $B_1$,
 \ldots, $B_k$. 
\item \emph{Egypagáció} (\emph{simpagation}):\\
 $H_1$, \ldots, $H_l$ \cd{\bs} $H_{l+1}$, \ldots, $H_i$ \cd{==>} $G_1$, \ldots, $G_j$ \cd{|} $B_1$,
 \ldots, $B_k$. 
\end{itemize}

Amint látható, egy CHR szabály alapvetően három részre osztható:

\begin{itemize}
\item \emph{multi-fej} (\emph{multi-head}): ez alatt a ,,$H_1$, \ldots, $H_i$'' részt
értjük, ahol a $H_m$-ek CHR-korlátok
\item \emph{őr} (\emph{guard}): $G_1$, \ldots, $G_j$, ahol a $G_m$-ek a gazda-nyelvben
értelmezett korlátok
\item \emph{törzs} (\emph{body}): $B_1$, \ldots, $B_k$, ahol a $B_m$-ek CHR- vagy
gazda-korlátok
\item mindvégig $i > 0, j \geq 0, k \geq 0, l > 0$.
\end{itemize}

A fentiekben mindvégig $i > 0, j \geq 0, k \geq 0, l > 0$.
\br
Az \emph{egyszerűsítés} azt fejezi ki, hogy ha az őrben felírt korlátok igazak, akkor
a fej ekvivalens a törzzsel, ezért a fejet ki lehet törölni a korlát-tárból, és helyette
a törzset fel lehet venni. A \emph{propagáció} esetében ha az őr igaz, akkor a fejből
következik a törzs, ezért a törzset fel lehet venni a korlát-tárba. Az \emph{egypagáció}
az egyszerűsítés és a propagáció összegyúrásából keletkezik, és vissza is lehet vezetni
rájuk, hiszen: \cd{Heads1 \bs ~Heads2 <=> Body} ugyanazt jelenti, mint
\cd{Heads1, Heads2 <=> Heads1, Body}, csak sokkal hatékonyabb, mert \cd{Heads1}-et
nem kell újra felvenni a korlát-tárba.

\section{A CHR szabályok végrehajtása}

A CHR korlátoknak három állapota lehet: \emph{aktív}, \emph{aktiválható passzív} és
\emph{alvó passzív}. Aktív korlátból legfeljebb egy van minden pillanatban. Egy
korlát akkor válik aktiválhatóvá, ha valamelyik változóját \emph{megérintik}, azaz
egyesítik egy tőle különböző kifejezéssel. Minden alkalommal, amikor egy korlát aktívvá
válik, az összes rá vonatkozó szabályt végigpróbáljuk az alábbiak szerint:

\begin{itemize}
\item mindegyik fejre \emph{illesztjük} a korlátot (ez egyirányú egyesítés,
hívásbeli változó nem kaphat értéket)
\item többfejű szabályok esetén a korlát-tárban keresünk megfelelő
(illeszthető) \emph{partner}-korlátot,
\item sikeres illesztés után végrehajtjuk az őr-részt, ha ez is sikeres, a
szabály \emph{tüzel}, különben folytatjuk a próbálkozást a következő szabállyal.
\item A tüzelés abból áll, hogy (egyszerűsítés vagy egypagáció esetén)
kivesszük a tárból a kijelölt korlátokat, majd minden esetben végrehajtjuk a törzset.
\item Ha ezzel az aktív korlátot nem hagytuk el a tárból, folytatjuk a rá vonatkozó
próbálkozást a következő szabállyal.
\item Amikor az összes szabályt kipróbáltuk, akkor a korlátot \emph{elaltatjuk}, azaz
visszatesszük a tárba (az alvó passzív korlátok közé).
\end{itemize}

A futás akkor fejeződik be, amikor már nem marad aktiválható korlát. Az őr-részben
(elvben) nem lehet változót érinteni. Az őr-rész két komponensből áll: \cd{Ask \& Tell}

\begin{itemize}
\item \cd{Ask} --- változó-érintés vagy behelyettesítési hiba meghiúsulást okoz
\item \cd{Tell} --- nincs ellenőrzés, a rendszer ,,elhiszi'', hogy ilyen dolog nem
fordul elő
\end{itemize}

\section{A CHR szabályok szintaxisa}

A SICStus kézikönyv alapján a CHR szabályok szintaxisa az alábbi szabályrendszerrel
írható le:

\begin{alltt}
Szabály       --> [Név @] 
                    (Egyszerűsítés | Propagáció | Egypagáció)
                       [pragma Pragma].

Egyszerűsítés --> Fejek         <=> [Őr '|'] Törzs
Propagáció    --> Fejek         ==> [Őr '|'] Törzs
Egypagáció    --> Fejek \bs Fejek <=> [Őr '|'] Törzs

Fejek         --> Fej | Fej, Fejek
Fej           --> Korlát | Korlát \# Azonosító
Korlát        --> {\rm egy korlátként deklarált meghívható kifejezés}
Azonosító     --> {\rm egy egyedi változó}

Őr            --> Ask | Ask & Tell
Ask           --> Célsorozat
Tell          --> Célsorozat
Célsorozat    --> {\rm egy meghívható kifejezés, konjunkciókkal és diszjunkciókkal}

Törzs         --> Célsorozat

Pragma        --> {\rm kifejezések konjunkciója, amik a \cd{\#/2}-vel azonosított fejekre hivatkoznak}
\end{alltt}

Két fontosabb pragmát érdemes ismerni:

\begin{itemize}
\item \cd{already_in_heads(Id)} --- kiküszöböli ugyanazon korlát kivételét és
                   visszarakását
\item \cd{passive(Id)} --- a hivatkozott fej-korlát csak passzív szerepű lehet. 
\end{itemize}

\section{CHR példák}

Az alábbi példa az $X \leq Y$ relációt valósítja meg CHR-ben.
\begin{prologcode}
:- use_module(library(chr)).

handler leq.
constraints leq/2.

:- op(500, xfx, leq).

reflexivity  @ X leq Y <=> X = Y | true.
antisymmetry @ X leq Y , Y leq X <=> X=Y.
idempotence  @ X leq Y \ X leq Y <=> true.
transitivity @ X leq Y , Y leq Z ==> X leq Z.

| ?- X leq Y, Y leq Z, Z leq X.

% X leq Y, Y leq Z  ----> (transitivity) X leq Z
% X leq Z, Z leq X  <---> (antisymmetry) X = Z
% Z leq Y, Y leq Z  <---> (antisymmetry) Z = Y

Y = X, Z = X ? 
\end{prologcode}

A $\leq$ reláció leírásához négy szabályt vezetünk be, mindegyik egy-egy elég
triviális tulajdonságot ír le:

\begin{itemize}
\item Reflexivitás (\cd{reflexivity}) --- kifejezi, hogy ha a korlát-tárban van
egy \cd{X leq Y} korlát, és ugyanekkor az \cd{X = Y} reláció teljesül, akkor
a korlát triviálisan igaz, tehát helyettesíthetjük a \cd{true} korláttal, azaz
tulajdonképpen eltávolíthatjuk a korlát-tárból.

\item Antiszimmetria (\cd{antisymmetry}) --- kifejezi, hogy ha a korlát-tárban
egyszerre van jelen az \cd{X leq Y} és az \cd{Y leq X} korlát, akkor az csak úgy
lehetséges, ha \cd{X=Y}.

\item Idempotencia (\cd{idempotency}) --- kifejezi, hogy ha a korlát-tárban
kétszer szerepel az \cd{X leq Y} korlát, akkor az egyiket eltávolíthatjuk.

\item Tranzitivitás (\cd{transitivity}) --- az
$X \leq Y \wedge Y \leq Z \Longrightarrow X \leq Z$ azonosság kifejezése.
\end{itemize}

Ezen négy szabály segítségével az \cd{X leq Y, Y leq Z, Z leq X} célsorozatból
a következőképpen jön rá a rendszer, hogy \cd{X = Y} és \cd{X = Z}:

\begin{center} \begin{tabular}{|l|l|l|}
\hline
Korlát-tár                 & Célsorozat                     & Tüzelés \\
\hline
\hline
---                        & \cd{X leq Y, Y leq Z, Z leq X} & ---     \\
\hline
\cd{X leq Y}               & \cd{Y leq Z, Z leq X}          & ---     \\
\hline
\cd{X leq Y, Y leq Z}      & \cd{Z leq X} & \cd{transitivity} $\Longrightarrow$ \cd{X leq Z} \\
\hline
\cd{X leq Y, Y leq Z, X leq Z} & \cd{Z leq X}               & ---     \\
\hline
\cd{X leq Y, Y leq Z, X leq Z} & & \cd{antisymmetry} $\Longrightarrow$ \cd{Z = X}\\
\cd{Z leq X} & & \\
\hline
\cd{X leq Y, Y leq X, X leq X} & & \cd{reflexivity} kétszer \\
\cd{X leq X, Z = X}            & & \\
\hline
\cd{X leq Y, Y leq X, Z = X}   & & \cd{antisymmetry} $\Longrightarrow$ \cd{Y = X} \\
\hline
\cd{Y = X, Z = X} & & \\
\hline
\end{tabular} \end{center}

Nézzünk egy bonyolultabb CHR példát végeshalmaz-korlátokra! Az alábbi példa egy egyszerű
\clpfd keretrendszert valósít meg:

\begin{itemize}
\item két-argumentumú korlátokat kezel;
\item a korlátokat  egy (a keretrendszeren kívül megadott) {\tt test/3}
eljárás írja le:
\begin{quote}
{\tt test(C, X, Y)} sikeres, ha a {\tt C} ,,nevű'' korlát fennáll {\tt X}
és {\tt Y} között;
\end{quote}
\item nem csak numerikus tartományokra jó.
\end{itemize}

\begin{prologcode}
handler dom_consistency.
constraints dom/2, con/3.
% dom(X,D) - az X változó a D listából veheti az értékeit
%            (D elemei ground-ok)
% con(C,X,Y) - a C korlát az X és Y változók között fennáll

con(C, X, Y) <=> ground(X), ground(Y) | test(C, X, Y).
con(C, X, Y), dom(X, XD) \ dom(Y, YD) <=> 
        reduce(x_y, XD, YD, C, NYD) | new_dom(NYD, Y).
con(C, X, Y), dom(Y, YD) \ dom(X, XD) <=> 
        reduce(y_x, YD, XD, C, NXD) | new_dom(NXD, X).

  reduce(CXY, XD, YD, C, NYD):- 
        select(GY, YD, NYD1), % megpróbálja YD-t csökkenteni GY-nal
        (   member(GX, XD), test(CXY, C, GX, GY) -> fail
        ;   reduce(CXY, XD, NYD1, C, NYD) -> true
        ;   NYD = NYD1
        ), !.

  test(x_y, C, GX, GY):- test(C, GX, GY).
  test(y_x, C, GX, GY):- test(C, GY, GX).

  new_dom([], _X) :- !, fail.
  new_dom(DX, X):- dom(X, DX),
          (   DX = [E] -> X = E
          ;   true
          ).

% labeling:
constraints labeling/0.

labeling, dom(X, L) #Id <=> member(X, L), labeling
        pragma passive(Id).
\end{prologcode}

A \cd{passive(Id)} úgynevezett \cd{pragma} direktíva jelentése: az \cd{\#Id}
jelöléssel hivatkozott fej csak passzív szerepű lehet. A fenti keretrendszer
alkalmazásával az N királynő példa megoldása:

\begin{prologcode}

% Qs az N-királynő feladat megoldása
queens(N, Qs) :-
        length(Qs, N), 
        make_list(1, N, L1_N), 
        domains(Qs, L1_N),       % tartományok megadása
        safe(Qs),                % korlátok felvétele
        labeling.                % címkézés

% make_list(I, N, L): Az L lista az I, I+1, ..., N elemekből áll.
make_list(I, N, []) :- I > N, !.
make_list(I, N, [I|L]) :-
        I1 is I+1,
        make_list(I1, N, L).

% domains(Vs, Dom): A Vs-beli változók tartománya Dom.
domains([], _).
domains([V|Vs], Dom) :- dom(V, Dom), domains(Vs, Dom).

% queens(Qs): Qs egy biztonságos királynő-elrendezés.
safe([]).
safe([Q|Qs]) :- no_attack(Qs, Q, 1), safe(Qs).

% no_attack(Qs, Q, I): A Qs lista által leírt királynők
% egyike sem támadja a Q által leírt királynőt, ahol I a Qs
% lista első elemének távolsága Q-tól.
no_attack([], _, _).
no_attack([X|Xs], Y, I) :-
        con(no_threat(I), X, Y),  % a korlát felvétele
        I1 is I+1, 
        no_attack(Xs, Y, I1).

% "Az X és Y oszlopokban  I sortávolságra levő királynők nem 
% támadják egymást" korlát definíciója, a dom_consistency 
% keretrendszernek megfelelően
test(no_threat(I), X, Y) :-
        Y =\= X, Y =\= X-I, Y =\= X+I.

| ?- queens(4, Qs).
                                Qs = [3,1,4,2], labeling ? ;
                                Qs = [2,4,1,3], labeling ? ; no
\end{prologcode}

Egy nem korlát-jellegű példa: az eratoszthenészi prímszita CHR-ben:

\begin{prologcode}
handler eratosthenes.
constraints primes/1,prime/1.

primes(1) <=> true.
primes(N) <=> N>1 | 
        M is N-1,prime(N),primes(M). 

absorb(J) @ prime(I) \ prime(J) <=> 
        J mod I =:= 0 | true.
\end{prologcode}

Boole-korlátok (ld. még \cd{library('chr/examples/bool.pl')}):

\begin{prologcode}
handler bool.
constraints and/3, labeling/0. 

and(0,X,Y) <=> Y=0.                      
and(X,0,Y) <=> Y=0.                      
and(1,X,Y) <=> Y=X.                      
and(X,1,Y) <=> Y=X.                      
and(X,Y,1) <=> X=1,Y=1.                  
and(X,X,Z) <=> X=Z.                      
and(X,Y,A) \ and(X,Y,B) <=> A=B.         
and(X,Y,A) \ and(Y,X,B) <=> A=B.         
                                      
labeling, and(A,B,C)#Pc <=> 
        label_and(A,B,C), labeling               
    pragma passive(Pc).        
                               
label_and(0,_X,0).      
label_and(1,X,X).      

| ?- and(X, Y, 0), labeling.   
  X = 0, labeling ? ;          
  X = 1, Y = 0, labeling ? ;
  no
\end{prologcode}

Számosság-korlát megvalósítása 0-1 értékű változókra:

\begin{prologcode}
constraints card/4.

% L-ben a 1-ek száma >= A és =< B.
card(A, B, L):-  
        length(L,N), A=<B,0=<B,A=<N, card(A,B,L,N).

triv_sat @ card(A,B,L,N) <=> A=<0,N=<B | true.          
pos_sat @ card(N,B,L,N) <=> set_to_ones(L).             
neg_sat @ card(A,0,L,N) <=> set_to_zeros(L).            
pos_red @ card(A,B,L,N) <=> select(X,L,L1),X==1 |       
                A1 is A-1, B1 is B-1, N1 is N-1, 
                card(A1,B1,L1,N1).
neg_red @ card(A,B,L,N) <=> select(X,L,L1),X==0 | 
                N1 is N-1, card(A,B,L1,N1).
% speciális esetek két változóra
card2nand @ card(0,1,[X,Y],2) <=> and(X,Y,0).           
% ...
labeling, card(A,B,L,N)#Pc <=> 
  label_card(A,B,L,N), labeling
    pragma passive(Pc).

label_card(A,B,[],0):- A=<0,0=<B.
label_card(A,B,[0|L],N):- N1 is N-1, card(A,B,L,N1).
label_card(A,B,[1|L],N):- 
    A1 is A-1, B1 is B-1, N1 is N-1, card(A1,B1,L,N1).

| ?- card(2,3,L), labeling.

L = [1,1], labeling ? ;      
L = [0,1,1] , labeling ? ; 
L = [1,0,1] , labeling ? ;   
L = [1,1,_A] , labeling ? ; 
L = [0,0,1,1] , labeling ? ; 
L = [0,1,0,1] , labeling ? ; 
L = [0,1,1,_A] , labeling ? ;  
% ...
\end{prologcode}

\section{Egy nagyobb CHR példa kezdeménye}

A feladat: adott egy négyzet, ahol bizonyos mezőkben egész számok vannak.
A cél: minden mezőbe számot írni, úgy, hogy az azonos számot tartalmazó
összefüggő területek mérete megegyezzék a terület mezőibe írt számmal. 
\br
A feladványt leíró adatstruktúra: \cd{tf(Meret,Adottak)}, ahol \cd{Meret}
a négyzet oldalhossza, az \cd{Adottak} egy lista, amelynek elemei 
\cd{t(O,S,M)} alakú struktúrák. Egy ilyen struktúra azt jelenti, hogy a
négyzet \cd{S}. sorának \cd{O}. oszlopában az \cd{M} szám áll.

\begin{prologcode}
handler terulet.
constraints orszag/3, tabla/1, cimkez/0.

% orszag(Mezok, M, N): A Mezok mezőlista egy összefüggő, M méretű
% terület, amelynek kívánt mérete N. Egy mező Sor-Oszlop
% koordinátáival van megadva.

% tabla(Matrix): A teljes téglalap, listák listájaként.

% cimkez: Címkézési segédkorlát (ld. labeling az előző példákban)

foglalas(tf(Meret,Adottak), Mtx) :-
    bagof(Sor,
          S^bagof(Mezo, 
                  O^tabla_mezo(Meret, Adottak, S, O, Mezo), 
                  Sor),
          Mtx),
    append_lists(Mtx, Valtozok),         % listává lapítja Mtx-t
    MaxTerulet is Meret*Meret,
    domain(Valtozok, 1, MaxTerulet),
    tabla(Mtx),                          % tabla/1 korlát felvétele
    matrix_korlatok(Mtx, 1),             % orszag/3 korlátok
    cimkez.                              % címkézési segédkorlát

tabla_mezo(Meret, Adottak, S, O, M) :-
    between(1, Meret, S),               % 1..Meret felsorolása
    between(1, Meret, O),
    (   member(t(S,O,M), Adottak) -> true
    ;   true
    ).

matrix_korlatok([], _).
matrix_korlatok([Sor|Mtx], S) :-
    sor_korlatok(Sor, S, 1),
    S1 is S+1,
    matrix_korlatok(Mtx, S1).

sor_korlatok([], _, _).
sor_korlatok([M|Mk], S, O) :-
    orszag([S-O], 1, M),
    O1 is O+1,
    sor_korlatok(Mk, S, O1).


% Az orszag/3 korlátra vonatkozó szabályok

% Ha két ország szomszédos, és azonos számot tartalmaz, akkor
% össze kell vonni őket
orszag(Mezok1, H1, M), orszag(Mezok2, H2, M) <=>
                szomszedos_orszag(Mezok1, Mezok2) |
                H is H1+H2,
                M #>= H,
                append(Mezok1, Mezok2, Mezok),
                orszag(Mezok, H, M).

% Ha két ország szomszédos, és az egyik már pont annyi mezőből áll,
% ahányas számot tartalmaz, akkor a másik országban már nem szerepelhet
% ilyen szám
orszag(Mezok, M, M), orszag(Mezok1, _, M1) ==>
                szomszedos_orszag(Mezok, Mezok1) |
                M1 #\= M.

% Ha egy ország pont annyi mezőből áll, ahányas számot tartalmaz,
% akkor nem kell vele tovább foglalkozni (mivel a szabályok sorban
% tüzelnek, ezért ez mindig később tüzel, mint az előző)
orszag(Mezok, M, M) <=> 
                true.

% Ha egy ország már nem terjeszkedhet, de kevesebb mezőből áll, mint
% ahányas szám van benne, akkor meghiúsulunk
orszag(Mezok, H, M), tabla(Mtx) ==>
                nonvar(M), H < M, 
                \+ terjeszkedhet(Mezok, M, Mtx) | fail.

% Címkézési segédszabály
(orszag(Mezok, H, M) # Id1, tabla(Mtx) # Id2) \ cimkez  <=>
                fd_max(M, Max), H < Max | 
                szomszedos_mezo(Mezok, Mtx, M), cimkez
                        pragma passive(Id1), passive(Id2).

% A Mezok mezőből álló, M méretű ország az Mtx táblában terjeszkedhet
terjeszkedhet(Mezok, M, Mtx) :-
    szomszedos_mezo(Mezok, Mtx, M0),
    fd_set(M0, Set), fdset_member(M, Set).

% Az Mk1 és az Mk2 mezőkből álló országok szomszédosak
szomszedos_orszag(Mk1, Mk2) :-
    member(S1-O1, Mk1), member(S2-O2, Mk2),
    (   S1 == S2 -> abs(O1-O2) =:= 1
    ;   O1 == O2, abs(S1-S2) =:= 1
    ).

% A Mezok országnak az Mtx mátrixban az M mező a szomszédja
szomszedos_mezo(Mezok, Mtx, M) :-
    member(S-O, Mezok),
    relativ_szomszed(S1, O1),
    S2 is S+S1, O2 is O+O1,
    non_member(S2-O2, Mezok),
    matrix_elem(S2, O2, Mtx, M).   
    % A Mtx mátrix S2. sorának O2. eleme M.

relativ_szomszed(1, 0).
relativ_szomszed(0, -1).
relativ_szomszed(-1, 0).
relativ_szomszed(0, 1).

pelda(p1,  tf(5, [t(2,1,2),t(2,2,1),t(2,4,4),t(2,5,3),
                  t(3,4,2),t(4,2,5),t(4,4,3),t(5,1,3),
                  t(5,5,2)])).                        

pelda(p9,  tf(6, [t(1,1,1),t(2,3,1),t(2,6,4),t(3,1,3),t(3,6,3),
                  t(4,1,2),t(4,5,2),t(4,6,4),t(5,3,3),t(6,1,2),
                  t(6,5,3)])).                                 

| ?- pelda(p1, _Fogl), foglalas(_Fogl, Mtx).
Mtx = [[2,4,4,3,3],
       [2,1,4,4,3],
       [3,5,5,2,2],
       [3,5,3,3,3],
      [3,5,5,2,2]],
cimkez,
tabla([[2,4,4,3,3],[2,1,4,4,3],[3,5,5,2,2],...]) ? ;
no
\end{prologcode}
