\clearpage

\chapter{A Mercury nagyhatékonyságú LP megvalósítás\kieg}

Ebben a fejezetben a Mercury nagyhatékonyságú logikai programnyelvet mutatjuk be nagy
vonalakban. Mivel a Prolog önmagában nem kimondottan alkalmas nagyobb méretű projektek
kezelésére, a Mercury megalkotásakor a készítők célja elsődlegesen a nagybani programozás
támogatása, valamint a produktivitás, a megbízhatóság és a hatékonyság növelése volt.
Eközben a következő irányelveket tartották szem előtt:

\begin{itemize}
\item Teljesen deklaratív programozás
\item Funkcionális elemek integrálása
\item Hagyományos Prolog szintaxis megőrzése
\item Típus, mód és determinizmus információk használata
\item Szeparált fordítás támogatása
\item Prologénál erősebb modul-rendszer
\item Sztenderd könyvtár
\end{itemize}

A Mercury nyelv és az implementáció fejlesztője a University Of Melbourne. A nyelv
honlapja a \verb'http://www.cs.mu.oz.au/mercury/' címen található meg.

\section{Egy Mercury példaprogram}

A feladat: operációs rendszerek file-név-illesztéséhez hasonló
funkció megvalósítása. Egy karaktersorozat illesztésekor a \cd{?} karakter
egy tetszőleges másik karakterrel illeszthető, a \cd{*} karakter egy tetszőleges
(esetleg üres) karaktersorozattal illeszthető, a \verb+\+$c$ karakter-pár pedig
a $c$ karakterrel illeszthető. Ha egy minta \verb+\+-re végződik, az illesztés
meghiúsul. Bármely más karakter csak önmagával illeszthető.
\br
A program hívási formája: \cd{match Pattern1 Name Pattern2}, ahol a \cd{Pattern1}
és \cd{Pattern2} mintákban a \cd{*} és \cd{?} karaktereknek azonos elrendezésben
kell előfordulniuk.
\br
A program funkciója: a \cd{Pattern1} mintára az összes lehetséges módon illeszti
a \cd{Name} nevet, a \cd{*} és \cd{?} karakterek helyére kerülő szöveget \cd{Pattern2}-be
behelyettesíti, és az így kapott neveket kiírja.

\begin{verbatim}
:- module match.
/*-----------------------------------------------------*/
:- interface.

:- import_module io.
:- pred main(io__state::di, io__state::uo) is det. % kötelező

/*-----------------------------------------------------*/
:- implementation.
:- import_module list, std_util, string, char.

main --> 
    command_line_arguments(Args),
    (   {Args = [P1,N1,P2]} ->
         {solutions(match(P1, N1, P2), Sols)},
         format("Pattern `%s' matches `%s' as `%s' matches the following:\n\n",
                         [s(P1),s(N1),s(P2)]),
         write_list(Sols, "\n", write_string),
         write_string("\n*** No (more) solutions\n")
    ;   write_string("Usage: match <p1> <n1> <p2>\n")
    ).

:- pred match(string::in, string::in, string::in,
               string::out) is nondet. % szükséges
match(Pattern1, Name1, Pattern2, Name2) :-
    to_char_list(Pattern1, Ps1),
    to_char_list(Name1, Cs1),
    to_char_list(Pattern2, Ps2),
    match_list(Ps1, Cs1, L),
    match_list(Ps2, Cs2, L),
    from_char_list(Cs2, Name2).

:- type subst ---> any(list(char)) ; one(char).

:- pred match_list(list(char), list(char), list(subst)).
:- mode match_list(in, in, out) is nondet. % mindkettő,
:- mode match_list(in, out, in) is nondet. % vagy egyik se
match_list([], [], []).
match_list([?|Ps], [X|Cs], [one(X)|L]) :-
    match_list(Ps, Cs, L).
match_list([*|Ps], Cs, [any(X)|L]) :-
    append(X, Cs1, Cs),
    match_list(Ps, Cs1, L).
match_list([\, C|Ps],  [C|Cs], L) :-
    match_list(Ps, Cs, L).
match_list([C|Ps], [C|Cs], L) :-
    C \= (*), C \= ?, C \= (\), 
    match_list(Ps, Cs, L).
\end{verbatim}

\enumhead{A program egy futása}
\begin{verbatim}
# ./match **z?c  foozkc '|*|*|?'
Pattern '**z?c' matches 'foozkc' as '|*|*|?' matches the following:

    |foo||k
    |fo|o|k
    |f|oo|k
    ||foo|k

*** No (more) solutions
\end{verbatim}

\section{A Mercury modul-rendszere}

A Mercury programokat különálló modulokból lehet felépíteni, ez lehetővé teszi a
modulok egymástól független, szeparált fordítását. A modul-rendszer támogatja az
absztrakt típusok használatát és a modulok egymásba ágyazhatóságát is.
\br
Egy Mercury modult mindig a \cd{:- module} \meta{modulename}\cd{.} sorral kezdjük
(a fenti példában \cd{:- module match.}). A modul két részre osztható, az
\emph{interfész} és az \emph{implementációs} részre. Az interfész részt az
\cd{:- interface.} kulcsszóval kezdjük. Ebben a részben minden szerepelhet, kivéve
függvények, predikátumok és almodulok definícióját. Az itt lévő dolgok fognak
,,kilátszani'' a modulból. Az implementációs részt az \cd{:- implementation.} kulcsszó
vezeti be, és ide kerülnek az interfész részben deklarált függvények, predikátumok,
absztrakt adattípusok és almodulok pontos definíciói. Az implementációs rész tartalma
lokális a modulra nézve.
\br
Egy modul természetesen felhasználhatja egy másik modul interfész részét. Ehhez
a használni kívánt modult importálni kell az \cd{:- import_module} \meta{modules}\cd{.}
vagy az \cd{:- use_module} \meta{modules}\cd{.} predikátum használatával. A kettő
között az a lényegi különbség, hogy az \cd{import_module} alkalmazásával importált
modulokból származó predikátumok használatához nincs szükség külön modul-kvalifikációra,
míg a \cd{use_module} esetében igen. A modul-kvalifikáció alakja:
\meta{module}\cd{:}\meta{submodule}\cd{:}\dots\cd{:}\meta{submodule}\cd{:}\meta{name}.
A Mercury jelenlegi verzióiban a \cd{:} helyett egyelőre a \cd{\_\_} (dupla aláhúzásjel)
javasolt, mert lehet, hogy a későbbiekben a \cd{:} helyett a \cd{.} lesz a
modulkvalifikátor és a \cd{:} a típuskvalifikátor. Modulkvalifikációra egyébként a
mintaillesztő program interfész részében a
\cd{:- pred main(io__state::di, io__state::uo) is det.} sorban láthatunk egy egyszerű példát.
\br
Almodulok használata elvileg kétféleképpen is lehetséges: a főmodul fájljába beágyazva
(\emph{beágyazott almodul}) vagy külön fájlban (\emph{szeparált almodul}). A jelenlegi
implementációban a beágyazott almodulok használata egyelőre problémás.

\section{A Mercury típusrendszere}

A Mercury a Prologtól eltérően egy típusos nyelv. A típusoknak több fajtája lehet:

\begin{itemize}
\item primitív: \cd{char}, \cd{int}, \cd{float}, \cd{string}
\item predikátum: \cd{pred}, \cd{pred(T)}, \cd{pred(T1, T2)}, \dots
\item függvény: \cd{(func) = T}, \cd{func(T1) = T}, \dots
\item univerzális: \cd{univ}
\item ,,a világ állapota'': \cd{io\_\_state}
\item felhasználó által bevezetett
\end{itemize}

A felhasználói adattípusok között szerepel az SML-ből ismerős \emph{megkülönböztetett
unió} adattípus (SML-ben ez \cd{datatype} néven szerepelt), az \emph{ekvivalencia típus}
(más néven típusátnevezés, SML-ben \cd{type} néven szerepelt) és az
\emph{absztrakt típus} is. A megkülönböztetett uniónak is több alfajtája van:

\begin{itemize}
\item \emph{Enumeráció típus} \\
\cd{:- type fruit ---> apple; orange; banana; pear.} \\
A fenti példában egy \cd{fruit} nevű típust definiálunk, amely változói négy lehetséges
értéket vehetnek fel, és ezeket az értékeket rendre az \cd{apple}, \cd{orange},
\cd{banana}, \cd{pear} névkonstansokkal azonosítjuk.
\item \emph{Rekord típus} \\
\cd{:- type itree ---> empty; leaf(int); branch(itree, itree)} \\
Ez a példa egy \cd{itree} nevű típust definiál, amely bináris fák reprezentálására
szolgál. Egy \cd{itree} típusú változó értéke háromféle lehet: \cd{empty} (ez
jelképezi az üres fát), \cd{leaf(int)} (ez jelképez egy levelet, a levél értékének
típusa \cd{int}) és \cd{branch(itree, itree)} (ez egy elágazó csomópontot jelképez,
ahol mindkét ág egy-egy újabb \cd{itree} típusú objektum).
\item \emph{Polimorfikus típus} \\
\begin{verbatim}
:- type list(T) ---> [] ; [T|list(T)].
:- type pair(T1, T2) ---> T1 - T2.
\end{verbatim}
A \cd{list(T)} típus egy \cd{T} típusú elemekből álló lista, ahol \cd{T} maga
egy \emph{típusváltozó}. Így például a \cd{list(int)} \cd{int} típusú elemek
listáját adja, \cd{list(char)} karakterlistát, és így tovább. A \cd{pair(T1, T2)}
típus \cd{T1} és \cd{T2} típusú elemekből álló párok típusa, ahol \cd{T1} és
\cd{T2} szintén típusváltozók. Például \cd{pair(int,char)} egy olyan párt ír
le, amely első tagja \cd{int}, második tagja \cd{char} típusú.
\end{itemize}

A megkülönböztetett unió leírásának megalkotására vonatkozó általános szabályok:

\begin{itemize}
\item A típusleírás alakja: \cd{:- type} \meta{típus} \verb`--->` \meta{törzs}\cd{.}
\item a \meta{törzs} minden konstruktorában az argumentumok típusok vagy változók
\item a \meta{törzs} minden változójának szerepelnie kell \meta{típus}-ban
\item \meta{típus} változói különbözők
\item a típusok között névekvivalencia van
\item egy típusban nem fordulhat elő egynél többször azonos nevű és
argumentumszámú konstruktor
\end{itemize}

Az ekvivalencia típus leírása:

\begin{itemize}
\item \cd{:- type} \meta{típus} \cd{==} \meta{típus}\cd{.} (például
\cd{:- type assoc\_list(K, V) == list(pair(K, V)).})
\item nem lehet ciklikus
\item a jobb és a bal oldal teljesen ekvivalens egymással
\end{itemize}

Az absztrakt típusra az a jellemző, hogy az interfész részben csak a típus neve
van megadva (\cd{:- type} \meta{típus}\cd{.}), a típus tényleges definíciója az
implementációs részben el van rejtve.
\br
A típusoknak a predikátum- és függvényleírásoknál van szerepe: egy predikátum
vagy függvény leírásánál mindig meg kell adni a paraméterek típusát. Például:

\begin{verbatim}
:- pred is_all_uppercase(string).
:- func length(list(T)) = int.
\end{verbatim}

\section{Módok és behelyettesítettség}

\definicio egy paraméter \emph{mód}jának nevezünk egy két behelyettesítettségi állapotból
álló párt, ahol az első állapot azt jelöli, ahogyan a paraméter bemegy egy adott
predikátumba, a második pedig azt, ahogy kijön belőle. Például az \cd{out} mód jelentése:
szabad változó megy be, tömör kifejezés jön ki.
\br
Saját adattípusainkhoz különböző, részleges behelyettesítettséget leíró behelyettesítettségi
állapotot is rendelhetünk. Tekintsük például a bináris fa adattípusát:

\epsfigure{instree}{0.4}

Az állapot leírásakor a típust tartalmazó (,,vagy'') csúcsokhoz rendelünk
behelyettesítettségi állapotot. A deklarációban a \cd{bound/1}, a \cd{free/0} és a
\cd{ground/0} funktorokat használhatjuk. Például:
\begin{verbatim}
:- inst bs = bound(empty; leaf(free); branch(bs,bs)).
\end{verbatim}
A fenti deklaráció alapján \cd{bs} behelyettesítettségű az olyan bináris fa, amely
vagy üres (\cd{empty}), vagy egy szabad változót tartalmazó levél (\cd{leaf(free)}),
vagy egy olyan elágazás, amely mindkét fele \cd{bs} behelyettesítettségű. Lehetőség
van parametrizált \cd{inst}-ek készítésére is:
\begin{verbatim}
:- inst bs(Inst) = bound(empty ; leaf(Inst) ; branch(bs(Inst),bs(Inst))).
:- inst listskel(Inst) = bound([] ; [Inst|listskel(Inst)]).
\end{verbatim}

Egy mód leírása a behelyettesítettségi állapotok alapján a következőképpen néz ki:
\cd{:- mode} \meta{m} \cd{==} \meta{inst1} \verb`>>` \meta{inst2}\cd{.}, ahol
\meta{m} a mód neve, \meta{inst1} és \meta{inst2} pedig behelyettesítettségi állapotok
nevei. Az \cd{in} és az \cd{out} mód leírása például így néz ki:

\begin{verbatim}
:- mode in == ground >> ground.
:- mode out == free >> ground.
\end{verbatim}

Lehetőség van módok átnevezésére is: \cd{:- mode} \meta{m1} \cd{==} \meta{m2}\cd{.}

\begin{verbatim}
:- mode (+) == in.
:- mode (-) == out.
\end{verbatim}

Természetesen a parametrizálás lehetősége a móddeklarációknál is adott:

\begin{verbatim}
:- mode in(Inst) == Inst -> Inst.
:- mode out(Inst) == free -> Inst.
\end{verbatim}

A módokat a predikátum-mód deklarációkban lehet hasznosítani. Itt egy predikátumról
azt írjuk le, hogy milyen mód-kombinációkban szabad használni a predikátum paramétereit.
Például a \cd{lists} könyvtár \cd{append/3} eljárásának mindhárom bemeneti paramétere
\cd{list(T)} típusú (ahol \cd{T} tetszőleges típusnév), és két módban használható.
Az egyik módban az első két paraméter bemeneti, a harmadik kimeneti, és így az
eljárás listák összefűzésére szolgál. A másik módban az első két paraméter kimeneti,
a harmadik bemeneti, és így az eljárás a harmadik listát az összes lehetséges módon
szétszedi. Ezt predikátum-mód deklarációval a következőképpen lehet leírni:

\begin{verbatim}
:- pred append(list(T), list(T), list(T)).
:- mode append(in, in, out).
:- mode append(out, out, in).
\end{verbatim}

Ha csak egyetlen mód van, akkor azt össze lehet vonni a \cd{pred} deklarációval:

\begin{verbatim}
:- pred append(list(T)::in, list(T)::in, list(T)::out).
\end{verbatim}

Amire még figyelni kell a predikátum-mód deklarációkkal kapcsolatban:

\begin{itemize}
\item \cd{free} változókat még egymással sem lehet összekapcsolni, ezért az
alábbi példa hibás:
\begin{verbatim}
:- mode append(in(listskel(free)),
                in(listskel(free)),
                out(listskel(free))).
\end{verbatim}
\item Ha egy predikátumnak nincs predikátum-mód deklarációja, akkor a
fordító kitalálja az összes szükségeset (\verb'--infer-all' kapcsoló), 
de függvényeknél ilyenkor felteszi, hogy minden argumentuma \cd{in}
és az eredménye \cd{out}.
\item A fordító átrendezi a hívásokat, hogy a mód korlátokat kielégítse, ha
ez nem megy, hibát jelez. (Jobbrekurzió!  Lásd a \cd{match_list/3}
\cd{append/3} hívását!) Az átrendezés nem okozhat problémát, mivel a Mercury
tisztán logikai nyelv, ezért a hívások sorrendje tetszőleges.
\item A megadottnál ,,jobban'' behelyettesített argumentumokat
egyesítésekkel kiküszöböli a fordító.  Ezeket a módokat le sem kell írni
(de érdemes lehet). Például \cd{:- mode append(in, out, in).} a szétszedő
\cd{append}-et fogja használni.
\item A jelenlegi implementáció nem kezeli a részlegesen behelyettesített
adatokat.
\end{itemize}

\section{Determinizmus}

A Mercury-ban lehetőség van arra is, hogy minden predikátum minden módjára
(azaz minden eljárásra) megadjuk, hogy hányféleképpen sikerülhet, és hogy
meghiúsulhat-e. Hatféle determinizmus kategória létezik:

\begin{center}
\begin{tabular}{|l|l|l|l|}
\hline
meghiúsulás \bs megoldások &    $0$          &   $1$       &     $> 1$\\
\hline
  nem           &\cd{erroneous}  &\cd{det}     &     \cd{multi}\\
\hline
  igen          &\cd{failure}    &\cd{semidet} &     \cd{nondet}\\
\hline
\end{tabular}
\end{center}

Egy determinizmus-deklaráció formailag majdnem megegyezik egy egyszerű predikátum-mód
deklarációval, a különbség mindössze annyi, hogy a deklaráció végén egy \cd{is} kulcsszó
után oda kell írni a megfelelő determinizmus kategóriát is. Például az \cd{append/3}-ra:

\begin{verbatim}
:- mode append(in, in, out) is det.
:- mode append(out, out, in) is multi.
:- mode append(in, in, in) is semidet.
\end{verbatim}

Ha csak egyetlen mód van, akkor azt össze lehet vonni a \cd{pred} deklarációval:

\begin{verbatim}
:- pred p(int::in) is det.
p(_).
\end{verbatim}

Felvetődik a kérdés, hogy mi értelme van a két ,,egzotikus'' determinizmusnak, a
\cd{failure}-nak és az \cd{erroneous}-nak. A \cd{failure} egy olyan predikátumot
jelent, amely soha nem ad megoldást, és mindig meghiúsul. Ilyen például a \cd{fail/0}
eljárás. \cd{erroneous} determinizmussal a \cd{require\_error/1} eljárás rendelkezik,
amely hibaüzenetek kijelzésére alkalmas (az \cd{erroneous} determinizmus elvileg egy
olyan eljárást jelent, amely soha nem ad megoldást, de nem is hiúsul meg). Ez a két
determinizmus például hibakezelésre alkalmazható:

\begin{verbatim}
:- mode append(out, out, out) is erroneous.
\end{verbatim}

Ehhez természetesen írni kell egy olyan klózt is, amely mindhárom paraméter
behelyettesítetlensége esetén a \cd{require\_error/1} használatával hibát jelez.
\br
Függvények esetén ha minden argumentum bemenő, akkor a determinizmusuk csak
\cd{det}, \cd{semidet}, \cd{erroneous} vagy \cd{failure} lehet, a többszörös
megoldásokat nem szabad megengednünk, mert ilyen esetben nem is beszélhetünk
függvényről. Például a \cd{between(in, in, out)} nem írható le függvény alakban.


\section{Magasabbrendű eljárások}

A hagyományos Prolog megvalósításban a \cd{call/1} eljárás segítségével lehetőség
volt arra, hogy egy eljárásból egy másik eljárást hívjunk meg. A Mercury-ban
bevezették a \cd{call/2}, \cd{call/3} \ldots eljárásokat is, amelyek segítségével
úgy hívhatunk meg egy eljárást, hogy annak paraméterlistáját a \cd{call}-ban
átadott paraméterekkel kiegészítjük. Például a \cd{call/4} definíciója Prologban:

\begin{verbatim}
% Pred az A, B és C utolsó argumentumokkal meghívva igaz.
call(Pred, A, B, C) :- 
    Pred =.. FArgs, 
    append(FArgs, [A,B,C], FArgs3), 
    Pred3 =.. FArgs3,
    call(Pred3).
\end{verbatim}

Az ilyen jellegű \cd{call} hívások segítségével lehetőség nyílik magasabb rendű
eljárások egyszerű megvalósítására. Például az SML-ből ismert \cd{map} funkció:

\begin{verbatim}
% map(Pred, Xs, Ys): Az Xs lista elemeire 
% a Pred transzformációt alkalmazva kapjuk az Ys listát.
:- pred map(pred(X, Y), list(X), list(Y)).
:- mode map(pred(in, out) is det, in, out) is det.
:- mode map(pred(in, out) is semidet, in, out) is semidet.
:- mode map(pred(in, out) is multi, in, out) is multi.
:- mode map(pred(in, out) is nondet, in, out) is nondet.
:- mode map(pred(in, in) is semidet, in, in) is semidet.
map(P, [H|T], [X|L]) :-
         call(P, H, X),
         map(P, T, L).
map(_, [], []).
\end{verbatim}

Itt \cd{Pred} egy olyan eljárás, amelynek két paramétere van, és amely a két
paraméter között egy transzformációt valósít meg. A \cd{map} eljárás módjai
és determinizmusa a \cd{Pred} eljárástól függ, hiszen ha \cd{Pred} mondjuk
\cd{pred(in, out) is multi}, akkor a \cd{map} második és harmadik paramétere
is rendre \cd{in} és \cd{out} lesz, \cd{map} determinizmusát pedig \cd{Pred}
determinizmusa fogja megszabni. Néhány (szám szerint 5) lehetőség leírása a
fenti programkódban is megtalálható. A \cd{map} használata:

\begin{verbatim}
:- import_module int.

:- pred negyzet(int::in, int::out) is det.
negyzet(X, X*X).

:- pred p(list(int)::out) is det.
p(L) :-
         map(negyzet, [1,2,3,4], L).
\end{verbatim}

Itt \cd{negyzet/2} egy olyan predikátum, amely a négyzetre emelés transzformációját
valósítja meg. Érdemes észrevenni, hogy a Mercury-ban kikerülhetjük az aritmetikai
számításoknál az \cd{is}-zel való vacakolást, hiszen mivel a \cd{negyzet/2}
predikátum-mód deklarációja tartalmazza, hogy a második paraméter is \cd{int},
ebből a Mercury már rájön, hogy az \cd{X*X} kifejezés nem egy \cd{'*'/2} struktúrát
jelent, hanem egy szorzást.
\br
Az SML-ből ismert $\lambda$ (\emph{lambda}) kifejezés Mercury-ban is használható,
így a fenti két predikátumot egyetlen predikátumba vonhatjuk össze:

\begin{verbatim} 
:- pred p1(list(int)::out) is det.
p1(L) :-
         map((pred(X::in, Y::out) is det :- Y = X*X), [1,2,3,4], L).
\end{verbatim}

Itt a négyzetre emelés predikátumát a \cd{map} hívásba ágyazottan egy névtelen
$\lambda$-függvény segítségével valósítjuk meg.
\br
Magasabbrendű kifejezéseket háromféleképpen hozhatunk létre:

\begin{itemize}
\item tegyük fel, hogy létezik
\begin{verbatim}
:- pred sum(list(int)::in, int::out) is det.
\end{verbatim}
\item $\lambda$-kifejezéssel:
\begin{verbatim}
X = (pred(Lst::in, Len::out) is det :- sum(Lst, Len))
\end{verbatim}
\item az eljárás nevét használva (a nevezett dolognak csak egyféle módja lehet
és nem lehet 0 aritású függvény):
\begin{verbatim}
Y = sum
\end{verbatim}
\end{itemize}

\cd{X} és \cd{Y} típusa a fenti példákban \cd{pred(list(int), int)}.
\br
Magasabbrendű függvények létrehozására is háromféle lehetőség van:

\begin{itemize}
\item ha adott
\begin{verbatim}
:- func mult_vec(int, list(int)) = list(int).
\end{verbatim}
\item $\lambda$-kifejezéssel:
\begin{verbatim}
X = (func(N, Lst) = NLst :- NLst = mult_vec(N, Lst))
Y = (func(N::in, Lst::in) = (NLst::out) is det
         :- NLst = mult_vec(N, Lst))
\end{verbatim}
\item a függvény nevét használva:
\begin{verbatim}
Z = mult_vec
\end{verbatim}
\end{itemize}


Az SML-ből ismert ,,curry''-zés mechanizmusa Mercury-ban is működik (kivéve
a beépített nyelvi konstruktorokra (pl. \cd{=}, \cd{\bs=}, \cd{call}, \cd{apply}),
ezeket nem lehet ,,curry''-zni:

\begin{itemize}
\item \cd{Sum123 = sum([1,2,3])}: \cd{Sum123} típusa \cd{pred(int)}
\item \cd{Double = mult_vec(2)}: \cd{Double} típusa \cd{func(list(int)) =
list(int)}
\end{itemize}

A DCG nyelvtanok leírásánál külön szintaxis van az olyan eljárásokra, amelyek
egy akkumulátor-párt is használnak:

\begin{verbatim}
Pred = (pred(Strings::in, Num::out, di, uo) is det -->
    io__write_string("The strings are: "),
    { list__length(Strings, Num) },
    io__write_strings(Strings),
    io__nl
)
\end{verbatim}

Magasabbrendű eljárásokat kétféleképpen hívhatunk meg:

\begin{itemize}
\item \cd{call(Closure, Arg$_1$, \dots, Arg$_n$)} ($n\geq 0$)--- a \cd{Closure}
argumentumlistája kiegészül az \cd{Arg$_1$, \dots, Arg$_n$} argumentumokkal,
és úgy hívódik meg.
\item \cd{solutions(match(P1, N1, P2), Sols)} --- összegyűjti a \cd{match(P1, N1, P2)}
hívás összes megoldását \cd{Sols}-ba. Természetesen \cd{match(P1, N1, P2)} helyett
tetszőleges más predikátum is használható.
\end{itemize}

Függvények meghívására az \cd{apply} használható:
\cd{apply(Closure2, Arg$_1$, \dots, Arg$_n$)} ($n\geq 0$) hívásakor a \cd{Closure2}
argumentumlistája kiegészül az \cd{Arg$_1$, \dots, Arg$_n$} argumentumokkal,
és úgy hívódik meg, \cd{apply} eredménye a hívás eredménye lesz. Használata:
\cd{List = apply(Double, [1,2,3])}.

A magasabbrendű kifejezések determinizmusa a módjuk része (és nem a típusuké).
Például:

\begin{verbatim}
:- pred map(pred(X, Y), list(X), list(Y)).
:- mode map(pred(in, out) is det, in, out) is det.
\end{verbatim}

\enumhead{Beépített behelyettesítettségek}
\begin{itemize}
\item Eljárások:\\
\cd{pred(}\meta{mode$_1$}\cd{,} \dots\cd{,} \meta{mode$_n$}\cd{) is}
\meta{determinism}, ahol $n\geq 0$
\item Függvények:\\
\cd{(func) =} \meta{mode} \cd{is} \meta{determinism}\\
\cd{func(}\meta{mode$_1$}\cd{,} \dots\cd{,} \meta{mode$_n$}\cd{) =}
\meta{mode} \cd{is} \meta{determinism}, ahol $n>0$
\end{itemize}

\enumhead{Beépített módok}
\begin{itemize}
\item A nevük megegyezik a behelyettesítettségek nevével, és a pár mindkét
tagja ugyanolyan, a névnek megfelelő behelyettesítettségű.
\item Egy lehetséges definíció lenne:
\begin{verbatim}
:- mode (pred(Inst) is Det) == in(pred(Inst) is Det).
\end{verbatim}
\end{itemize}

\enumhead{Amire figyelni kell}
\begin{itemize}
\item Magasabbrendű kimenő paraméter:
\begin{verbatim}
:- pred foo(pred(int)).
:- mode foo(free -> pred(out) is det) is det.
foo(sum([1,2,3])).
\end{verbatim}
\item Magasabbrendű kifejezések nem egyesíthetők:\\
\cd{foo((pred(X::out) is det :- X = 6))} hibás.
\end{itemize}

\section{Problémák a determinizmussal}

A determinizmus bevezetésével ésszerű néhány korlátozást bevezetni. Ilyen például az,
hogy \cd{det} vagy \cd{semidet} módú eljárásokból nem hívhatunk \cd{nondet} vagy
\cd{multi} eljárást, hiszen ezzel elrontjuk a determinizmust. A Mercury programoknál a
főprogramnak (\cd{main/2}) szükségszerűen \cd{det}-nek kell lennie, ezzel viszont
látszólag elvesztettük a \cd{nondet} és a \cd{multi} eljárások hívásának lehetőségét,
hiszen a főprogram \cd{det}, és belőle nem hívhatóak \cd{nondet} és \cd{multi} eljárások.
Ilyen eljárások hívásakor döntenünk kell:

\begin{itemize}
\item Ha az összes megoldást akarjuk, akkor a \cd{std\_util\_\_solutions/2} használatával
explicit módon meg kell kerestetnünk az összes megoldást. Ezzel a \cd{nondet} vagy
\cd{multi} eljárást \cd{det}-té tudjuk tenni, mert a \cd{std\_util\_\_solutions/2} mindig
sikerülni fog
\item Ha csak egy megoldás akarunk, és mindegy, hogy melyiket, akkor vagy azt csináljuk,
hogy az eljárás kimenő változóit nem használjuk fel (mert ekkor az első utáni megoldásokat
levágja a rendszer), vagy pedig az úgynevezett \emph{committed choice nondeterminism}
kihasználásával (\cd{cc\_nondet}, \cd{cc\_multi} determinizmus) determinizáljuk a
nemdeterminisztikus eljárásokat. A \emph{committed choice nondeterminism} mechanizmust
olyan helyeken használjuk, ahol biztosan nem lesz szükség több megoldásra. Fontos
megjegyezni, hogy IO műveletek csak \cd{det}, \cd{cc\_nondet} és \cd{cc\_multi}
eljárásokban használhatóak, mivel az IO műveleteket a visszalépés során nem lehet
visszavonni, ezért fontos, hogy csak olyan helyeken lehessen őket alkalmazni, ahol
biztosan nem lesz visszalépés.
\item Ha csak néhány megoldást akarunk, akkor a \cd{std\_util\_\_do\_while/4} eljárást
használhatjuk.
\end{itemize}

Arra az esetre, ha egy olyan eljárást akarunk meghívni, amelynek minden megoldása
ekvivalens, még nincs igazi megoldás. A tervekben a \cd{unique [X] goal(X)} szerkezet
szerepel, de egyelőre még a C interfésszel kell trükközni ilyen esetekben.
\br
Tekintsük például az alábbi feladatot: soroljuk fel egy halmaz összes részhalmazát,
és minden megoldást pontosan egyszer adjunk ki! Egy halmaz egy részhalmazának
kiválasztása nyilvánvalóan többféleképpen is sikerülhet, ezért valamilyen módon
(például a \cd{cc\_multi} determinizmus használatával) ki kell küszöbölni a
nemdeterminizmust:

\begin{verbatim}
:- module resze.

:- interface.
:- import_module io.

:- pred main(io__state::di, io__state::uo) is cc_multi.

:- implementation.
:- import_module int, set, list, std_util.

main -->
         read_int_listset(L, S),
         io__write_string("Set version:\n"),
         {std_util__unsorted_solutions(resze(S), P)},
         io__write_list(P, " ", io__write),
         io__write_string("\n\nList version:\n"),
         {std_util__unsorted_solutions(lresze(L), PL)},
         io__write_list(PL, " ", io__write), io__nl.

:- pred read_int_listset(list(int)::out, set(int)::out,
                      io__state::di, io__state::uo) is det.
read_int_listset(L, S) -->
         io__read(R),
         {   R = ok(L0) -> L = L0, set__list_to_set(L, S)
         ;   set__init(S), L = []
         }.
\end{verbatim}

A \cd{resze(S)} a halmazokat \cd{set} absztrakt adattípussal fogja ábrázolni,
az \cd{lresze(L)} pedig \cd{list} adattípussal. A kétféle megoldás:

\begin{verbatim}
:- pred resze(set(T)::in, set(T)::out) is multi.
resze(A, B) :-
         set__init(Fix),
         resze(A, B, Fix).

:- pred resze(set(T)::in, set(T)::out, set(T)::in) is multi.
resze(A, B, Fix) :-
         (   set__member(X, A)
         ->  set__delete(A, X, A1),
             (   resze(A1, B, Fix)
             ;   resze(A1, B, set__insert(Fix, X))
             )
         ;   B = Fix
         ).

:- pred lresze(list(T)::in, list(T)::out) is multi.
lresze(A, B) :-
         lresze(A, B, []).

:- pred lresze(list(T)::in, list(T)::out, list(T)::in) is multi.
lresze(A, B, Fix) :-
         (   A = [X|A1],
             (   lresze(A1, B, Fix)
             ;   lresze(A1, B, [X|Fix])
             )
         ;   A = [], B = Fix
         ).
\end{verbatim}

A \cd{set\_\_member/2} felsoroló jellege miatt nem teljesíti azt a feltételt, hogy
minden megoldás pontosan egyszer jelenjen meg, a lista fejének leválasztása viszont
(szemi)determinisztikus, ezért a \cd{set}-ekkel operáló verzió többször is kiad
egy megoldást, míg a listákkal operáló nem:

\begin{verbatim}
benko:~/mercury$ ls resze*
resze.m
benko:~/mercury$ mmake resze.dep
mmc --generate-dependencies     resze
benko:~/mercury$ mmake resze
rm -f resze.c
mmc --compile-to-c --grade asm_fast.gc resze.m > resze.err 2>&1
mgnuc --grade asm_fast.gc -c resze.c -o resze.o
c2init --grade asm_fast.gc resze.c > resze_init.c
mgnuc --grade asm_fast.gc -c resze_init.c -o resze_init.o
ml --grade asm_fast.gc -o resze resze_init.o resze.o     
benko:~/mercury$ ls resze*
resze    resze.d    resze.dv   resze.m  resze_init.c
resze.c  resze.dep  resze.err  resze.o  resze_init.o
benko:~/mercury$ ./resze
[1, 2].
Set version:
[1, 2] [2] [1] [] [1, 2] [1] [2] []

List version:
[2, 1] [1] [2] []
benko:~/mercury$
\end{verbatim}

Egy másik \cd{cc\_multi} példa az N királynő feladat megoldására (egyszerű
generate-and-test jellegű megoldás, ahol \cd{perm/2} generálja a királynők
elrendezéseit, \cd{safe/1} pedig ellenőriz):

\begin{verbatim}
:- module queens.

:- interface.
:- import_module list, int, io.

:- pred main(state::di, io__state::uo) is cc_multi.

:- implementation.

main -->
         (   {queen([1,2,3,4,5,6,7,8], Out)} -> write(Out)
         ;   write_string("No solution")
         ), nl.

:- pred queen(list(int)::in, list(int)::out) is nondet.
queen(Data, Out) :-
         perm(Data, Out),
         safe(Out).

:- pred safe(list(int)::in) is semidet.
safe([]).
safe([N|L]) :-
         nodiag(N, 1, L),
         safe(L).

:- pred nodiag(int::in, int::in, list(int)::in) is semidet.
nodiag(_, _, []).
nodiag(B, D, [N|L]) :-
         D \= N-B, D \= B-N,
         nodiag(B, D+1, L).
\end{verbatim}
